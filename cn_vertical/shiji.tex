\documentclass{guji}

\usepackage{enumitem} % For better list control if needed



% Set Book Name (formerly Banxin text) - appears in section 1 of the center column
\setBookName{欽定四庫全書}

% Set Chapter Title - appears in section 2 of the center column (between yuwei)
\setChapterTitle{史記\\目錄}

\begin{document}
\begin{guji-content}[template=sikuquanshu]
欽定四庫全書
\begin{itemize}
    \item 史記卷一
    \item % 这里必须有一个 \item 来承接内层的列表
    \begin{itemize}
        \item \GridTextbox[height=12, box-align=fill]{漢太史令}司馬遷\Space 撰
        \item \GridTextbox[height=12, box-align=fill]{宋中郎外兵曹參軍}\GridTextbox[height=3, box-align=fill]{裴駰}集解
        \item \GridTextbox[height=12, box-align=fill]{唐國子博士弘文館學士}司馬貞索隠
        \item \GridTextbox[height=12, box-align=fill]{唐諸王侍讀率府長史}張守節正義
    \end{itemize}
    \item 五帝本紀第一
\end{itemize}


\begin{GridParagraph}[indent=3]
\jiazhu{集解裴駰曰凡是徐氏義稱徐姓名以别之餘者悉是駰註解并集衆家義索隠紀者記也本其事而記之故曰本紀又紀理也絲縷有紀而帝王書稱紀者言為後代綱紀也正義鄭玄注中候勅省圖云徳合五帝坐星者稱帝又坤靈圖云徳配天地在正不在私曰帝按太史公依世本大戴禮以黄帝顓頊帝嚳唐堯虞舜為五帝譙周應劭宋均皆同而孔安國尚書序皇甫謐帝王世紀孫氏注世本並以伏犧神農黄帝為三皇少昊顓頊髙辛唐虞為五帝裴松之史目云天子稱本紀諸侯曰世家本者繫其本系故曰本紀者理也統理衆亊繫之年月名之曰紀第者次序之目一者舉數之由故曰五帝本紀第一禮云動則左史書之言則右史書之左陽故記動右隂故記言言為尚書事為春秋按春秋時置左右史故云史記}
\end{GridParagraph}

黄帝者\jiazhu{集解徐廣曰號有熊索隠按有土徳之瑞土色黄故稱黄帝猶神農火徳王而稱炎帝然也此以黄帝為五帝之首蓋依大戴禮五帝徳又譙周宋均亦以為然而孔安國皇甫謐帝王代紀及孫氏註系本並以伏犧神農黄帝為三皇少昊髙陽髙辛唐虞為五帝註號有熊者以其本是有熊國君之子故也都軒轅之丘因以為名又以為號又據左傳亦號帝鴻氏也正義輿地志云涿鹿本名彭城黄帝初都遷有熊也按黄帝有熊國君乃少典國君之次子號曰有熊氏又曰縉雲氏又曰帝鴻氏亦曰帝軒氏母曰附寶之祁野見大電繞北斗樞星感而懐孕二十四月而生黄帝於壽丘夀丘在魯東門之北今在兖州曲阜縣東北六里生日角龍顔有景雲之瑞以土徳王故曰黄帝封泰山禪亭亭亭亭在牟隂}少典之子\jiazhu{集解譙周曰有熊國君少典之子也皇甫謐曰有熊今河南新鄭是也索隠少典者諸侯國號非人名也又按國語云少典娶有蟜氏女而生炎帝然則炎帝亦少典之子炎黄二帝雖則承帝王代紀中間凡隔八帝五百餘年若以少典是其父名豈黄帝經五百餘年而始代炎帝後為天子乎何其年之長也又按秦本紀云顓頊氏之裔孫曰女脩吞玄鳥之卵而生大業大業娶少典氏而生栢翳明少典是國號非人名也黄帝者少典氏後代之子孫賈逵亦以左傳髙陽氏有才子八人亦謂其後代子孫而稱為子是也譙周字允南蜀人魏散騎常侍徵不拜此註所引者是其人所著古史考之說也皇甫謐字士安晉人號玄晏先生今所引者是其所作帝王世紀也}姓公孫名曰軒轅\jiazhu{索隠按皇甫謐云黄帝生於壽丘長於姬水因以為姓居軒轅之丘因以為名又以為號是本姓公孫長居姬水因改姓姬}生而神靈弱而能言\jiazhu{索隠弱謂幼弱時也蓋未合能言之時而黄帝即言所以為神異也潘岳有哀弱子篇其子未七旬曰弱正義言神異也易曰隂陽不測之謂神書曰人惟萬物之靈故謂之神靈也}幼而徇齊\jiazhu{集解徐廣曰墨子曰年踰十五則聰明心慮無不徇通矣駰案徇疾齊速也言聖徳幼而疾速也索隠斯文未明今案徇齊皆徳也書曰聰明齊聖左傳曰子雖齊聖齊謂聖徳齊肅又按孔子家語及大戴禮並作叡齊一本作慧齊叡慧皆智也太史公採大戴禮而為此紀今彼文無作徇者史記舊本亦有作濬齊蓋古字假借徇為濬濬深也義亦並通爾雅齊速俱訓為疾尚書大傳曰多聞而齊給鄭註云齊疾也今裴氏註云徇亦訓疾未見所出或當讀徇為迅迅於爾雅與齊俱訓疾則迅濬雖異字而音同也又爾雅曰宣徇遍也濬通也是遍之與通義亦相近言黄帝幼而才智周徧且辯給也故墨子亦云年踰五十則聰明心慮不徇通矣俗本作十五非是按謂年老踰五十不聰明何得云十五}長而敦敏成而聰明\jiazhu{正義成謂年二十冠成人也聰明聞見明辯也此以上至軒轅皆大戴禮文}軒轅之時神農氏世衰\jiazhu{集解皇甫謐曰易稱庖犧氏沒神農氏作是為炎帝班固曰教民耕農故號曰神農索隠世衰謂神農氏後代子孫道徳衰薄非指炎帝}

\end{guji-content}


\end{document}
