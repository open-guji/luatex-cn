\documentclass{guji}

\begin{document}

\section{Debug Mode OFF}
Normal vertical text:
\VerticalRTT[border=true]{
  这是一段普通的竖排文字,用于对比调试模式关闭时的效果。
}

\newpage
\section{Debug Mode ON (Global)}
\GujiDebugOn
Vertical text with global debug:
\VerticalRTT[border=true]{
  开启全局调试模式后,你应该能看到每个字符的蓝色辅助框。
  \GridTextbox[height=3, border=true]{
    这是文本框,应该有红色辅助框。
  }
}

\newpage
\GujiDebugOff
\section{Partial Debug}
\directlua{
  cn_vertical.debug.enabled = true
  cn_vertical.debug.show_grid = false
  cn_vertical.debug.show_boxes = true
}
Only boxes should be shown here:
\VerticalRTT[border=true]{
  只显示文本框的辅助线,不显示字符格。
  \GridTextbox[height=2]{
    红色框。
  }
}

\end{document}
