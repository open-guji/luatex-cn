\documentclass{article}
\usepackage[paperwidth=200mm, paperheight=150mm, margin=10mm]{geometry}
\usepackage{cn_vertical}
\usepackage{fontspec}

\setmainfont{SimSun} % Ensure a Chinese font is available

\begin{document}

\section*{Multi-page Support Test}

Testing multi-page splitting with \texttt{n-column=4} (so \texttt{page-columns=9} by default).
Each page should have content in columns 0, 1, 2, 3 (skip 4), 5, 6, 7, 8 (skip 9/reset).

\VerticalRTT[
  height=80mm,
  grid-width=1.5em,
  grid-height=1.2em,
  n-column=4,
  border=true,
  outer-border=true
]{
  这是第一页,排版应该覆盖前几列。

  继续填充文字以触发分页。

  段落1。

  段落2。
  
  段落3。
  
  段落4。
  
  段落5。
  
  段落6。
  
  段落7。
  
  段落8。
  
  段落9。
  
  段落10。
  
  这是新的一个大段落,希望能把内容挤到第二页。

  中国古代书籍排版通常每八列设一版心。


  在这里我们通过设置 n-column 为 4 来模拟较窄的版面。


  当列数达到 9 时,应该会自动开启一个新的 PDF 页面。


  继续填充:


  赵钱孙李,周吴郑王。冯陈诸卫,蒋沈韩杨。


  朱秦尤许,何吕施张。孔曹严华,金魏陶姜。
  戚谢邹喻,柏水窦章。云苏潘葛,奚范彭郎。
  鲁韦昌马,苗凤花方。俞任袁柳,酆鲍史唐。
  费廉岑薛,雷贺倪汤。滕殷罗毕,郝邬安常。
  
  更多文字:
  天地玄黄,宇宙洪荒。日月盈昃,辰宿列张。
  寒来暑往,秋收冬藏。闰余成岁,律吕调阳。
  云腾致雨,露结为霜。金生丽水,玉出昆冈。
  剑号巨阙,珠称夜光。果珍李柰,菜重芥姜。
}

\end{document}
