\documentclass{article}
\usepackage{fontspec}
\usepackage{expl3}

% --- expl3 代码块 ---
\ExplSyntaxOn

% 定义一个整数变量用于存储行高
\int_new:N \g_ancient_grid_height_int
\int_set:Nn \g_ancient_grid_height_int { 10 } % 默认 10 字

% 定义用户接口:设置每行字数
\NewDocumentCommand{\setgridheight}{ m }
  {
    \int_set:Nn \g_ancient_grid_height_int { #1 }
    % 将 TeX 的变量值同步给 Lua
    \lua_now:n { AncientBook.line_limit = \int_use:N \g_ancient_grid_height_int }
  }

\ExplSyntaxOff
% --------------------

\setmainfont{Source Han Serif SC}[RawFeature={+vert}] 
\directlua{dofile("ancient_vert.lua")}

\begin{document}
\pagestyle{empty}
\raggedright

\section*{测试 1:每行 5 个字}
\setgridheight{5}
预发其志,先格其物。古籍版式。

\vspace{10cm} % 留出空间观察

\section*{测试 2:每行 12 个字}
\setgridheight{12}
这段文字现在应该按照每行十二个字的规格进行垂直排版。
你可以看到,通过 expl3 我们实现了对底层 Lua 逻辑的动态控制。

\end{document}