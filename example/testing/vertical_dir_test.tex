% Test dir attribute for vertical text
% 测试使用 LuaTeX 的 dir 属性实现竖排
\documentclass[12pt]{article}

% Load Lua module with luatexbase support
\usepackage{luatexbase}
\directlua{dofile('../../cn_vertical/cn_vertical_dir.lua')}

\usepackage{fontspec}
\setmainfont{SimSun}

% Make page smaller for easier viewing
\usepackage[paperwidth=14cm,paperheight=20cm,margin=1cm]{geometry}

% Define test commands
\newcommand{\verticalA}[1]{%
  \par\noindent
  \addfontfeatures{RawFeature=+vert}%
  \directlua{cn_vertical_dir.vertical_simple([===[#1]===])}%
  \par
}

\newcommand{\verticalB}[1]{%
  \par\noindent
  \addfontfeatures{RawFeature=+vert}%
  \directlua{cn_vertical_dir.vertical_with_localdir([===[#1]===])}%
  \par
}

\begin{document}

\section*{横排对照 (Horizontal)}

这是正常横排:一二三四五六七八九十

\section*{方法 1: vbox 堆叠 (不使用 dir)}

说明:使用传统的 \texttt{\textbackslash vbox} 将每个字符包在 \texttt{\textbackslash hbox} 中堆叠。

\verticalA{一二三四五六七八九十}

\section*{方法 2: vbox + localdir (尝试 dir 属性)}

说明:与方法1相同,但准备添加 dir 属性支持。

\verticalB{一二三四五六七八九十}

\section*{对比说明}

\begin{itemize}
\item 方法 1 和方法 2 目前效果相同
\item 都是简单的字符从上到下排列
\item 字符保持正立(不旋转)
\item 符合传统竖排的汉字排列方式
\end{itemize}

\section*{关于 dir 属性的说明}

LuaTeX 的 \texttt{dir} 属性主要用于控制文本流动方向:
\begin{itemize}
\item \texttt{TLT}: 横排从左到右
\item \texttt{TRT}: 横排从右到左
\item \texttt{LTL}: 纵排,每列从左到右
\item \texttt{RTT}: 纵排,每列从右到左(传统竖排)
\end{itemize}

但是,对于简单的汉字竖排(只是字符从上到下排列),使用 \texttt{\textbackslash vbox} 堆叠已经足够,不需要复杂的 dir 属性操作。

\textbf{dir 属性的真正用途}是在更复杂的场景:
\begin{itemize}
\item 多列竖排(从右到左排列)
\item 处理混合文本(汉字 + 西文)
\item 处理标点符号的位置
\item 实现完整的页面级竖排布局
\end{itemize}

\section*{混合测试}

横排文字在前

\verticalA{竖排文字}

横排文字在后

\end{document}
