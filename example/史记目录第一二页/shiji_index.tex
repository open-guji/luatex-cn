\documentclass{guji}
\usepackage{enumitem} % For better list control if needed

% Define a custom template for another style (simulating another book)
\defineGujiTemplate{compact}{
    font-size = 24pt,
    line-spacing = 30pt,
    grid-width = 40pt,
    grid-height = 35pt,
    height = 500pt
}

% Set Book Name (formerly Banxin text) - appears in section 1 of the center column
\title{欽定四庫全書}

% Set Chapter Title - appears in section 2 of the center column (between yuwei)
\chapter{史記\\目錄}

\begin{document}
\begin{guji-content}[template=sikuquanshu]
欽定四庫全書\Space[5] 史部一
\begin{itemize}
    \item 史記目錄 正史類
    \item % 这里必须有一个 \item 来承接内层的列表
    \begin{itemize}
        \item \GridTextbox[height=12, box-align=fill]{漢太史令}司馬遷 撰
        \item 本紀一十二
        \item 年表一十
        \item 八書八
        \item 世家三十
        \item 列傳七十\EmptyBox[2]共一百三十卷
    \end{itemize}
\end{itemize}
本紀

史記卷一
\begin{itemize}
    \item 本紀第一
    \item \begin{itemize}
        \item 五帝
    \end{itemize}
\end{itemize}

史記卷二
\begin{itemize}
    \item 本紀第二
    \item \begin{itemize}
        \item 夏
    \end{itemize}
\end{itemize}

史記卷三
\begin{itemize}
    \item 本紀第三
    \item \begin{itemize}
        \item 殷
    \end{itemize}
\end{itemize}

史記卷四
\begin{itemize}
    \item 本紀第四
    \item \begin{itemize}
        \item 周
    \end{itemize}
\end{itemize}

史記卷五
\begin{itemize}
    \item 本紀第五
    \item \begin{itemize}
        \item 秦昭襄王
        \item 莊襄王
    \end{itemize}
\end{itemize}

史記卷六
\begin{itemize}
    \item 本紀第六
    \item \begin{itemize}
        \item 秦始皇帝
        \item 二世皇帝
    \end{itemize}
\end{itemize}

史記卷七
\begin{itemize}
    \item 本紀第七
    \item \begin{itemize}
        \item 項羽
    \end{itemize}
\end{itemize}

\end{guji-content}


\end{document}
