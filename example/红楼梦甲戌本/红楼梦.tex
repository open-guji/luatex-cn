\documentclass[红楼梦甲戌本]{ltc-guji}
\禁用分页裁剪

% === 字体设置示例 ===
% 用户可以在这里使用 fontspec 设置自己的字体
% 取消注释下面两行即可使用汲古书体:
\usepackage{fontspec}
\setmainfont{TW-Kai}[CharacterWidth=Full]
% 如果不设置,将使用系统默认字体

\title{石頭記}

\chapter{卷一}

\SetPageNumber{4}

\begin{document}
\begin{正文}
第一回

\begin{段落}[indent=2]
甄士隱夢幻識\夹注{通靈} 賈雨村風塵懷\夹注{閨秀}
\end{段落}

列位看官你道此書從何而來說起根由雖\SidePizhu{自占地步}近荒唐\SidePizhu{自首荒唐妙}細諳則深有趣味待在下將此來歷註明方使閱者了然不惑原來女媧氏煉石\SidePizhu{補天濟世勿認真用常言}補天之時于大\SidePizhu{荒唐也}荒山無稽崖煉成高經十二丈方經二十四丈頑石三萬六千五百零一塊媧皇氏只用了三萬六千五百塊只单单的剩了一塊未用便棄在此山青埂峰下誰知此石自經煆煉之後靈性已通因見衆石俱得補天獨自己無材不堪入選遂自怨自嘆日夜悲號慚愧一日正當嗟悼之際俄見一僧一道遠遠而來生得骨格不凡丰神迥別說說笑笑來至峰下坐于石邊高談快論先是說些雲山霧海神僲玄幻之事後便說到紅塵中榮華富貴此石聽了不覺打動凡心也想要到人間去享一享這榮華富貴但自恨粗蠢不得已便口吐人言向那僧道說道大師弟子蠢物不能見禮了適聞二位談那人世間榮耀繁華\SidePizhu{甲戌側批:紅塵中榮華富貴}心切慕之弟子質雖粗蠢性却稍通况見二師仙形道體定非凡品必有補天濟世之材利物濟人之德如蒙發一㸃慈心攜帶弟子得入紅塵在那富貴場中溫柔鄉裏受享幾年自當永佩洪恩萬劫不忘也二仙師聽畢齊憨笑道善哉善哉那紅塵中有却

\end{正文}
\end{document}