% 史记五帝本纪 - guji-digital 纯布局模式版本
% 使用 TextFlow 等 digital 命令,不使用 guji.cls 的语义命令(如 \夹注、\段落)
% 纯布局模式:每个换行符 = 一个新列

\documentclass[SiKuQuanShu-colored]{guji-digital}

\usepackage{tikz}
\setmainfont{TW-Kai}

\title{欽定四庫全書}
\chapter{史記\\卷一}

\begin{document}
\begin{数字化内容}
欽定四庫全書
\印章[page=1,opacity=0.7,color=black]{文渊阁宝印.png}%
\缩进[1] 史記卷一
\缩进[2] \填充文本框[12]{漢太史令}司馬遷\空格 撰
\缩进[2] \填充文本框[12]{宋中郎外兵曹參軍}\填充文本框[3]{裴駰}集解
\缩进[2] \填充文本框[12]{唐國子博士弘文館學士}司馬貞索隠
\缩进[2] \填充文本框[12]{唐諸王侍讀率府長史}張守节正义
\缩进[1] 五帝本紀第一
\缩进[3]\双列{\右小列{集解裴駰曰凡是徐氏義稱徐姓名以别之餘}\左小列{者悉是駰註解并集衆家義索隠紀者記也本}}
\缩进[3]\双列{\右小列{其事而記之故曰本紀又紀理也絲縷有紀而}\左小列{帝王書稱紀者言為後代綱紀也正義鄭玄注}}
\缩进[3]\双列{\右小列{中候勅省圖云徳合五帝坐星者稱帝又坤靈}\左小列{圖云徳配天地在正不在私曰帝按太史公依}}
\缩进[3]\双列{\右小列{世本大戴禮以黄帝顓頊帝嚳唐堯虞舜為五}\左小列{帝譙周應劭宋均皆同而孔安國尚書序皇甫}}
\缩进[3]\双列{\右小列{謐帝王世紀孫氏注世本並以伏犧神農黄帝}\左小列{為三皇少昊顓頊髙辛唐虞為五帝裴松之史}}
\缩进[3]\双列{\右小列{目云天子稱本紀諸侯曰世家本者繫其本系}\左小列{故曰本紀者理也統理衆亊繫之年月名之曰}}
\缩进[3]\双列{\右小列{紀第者次序之目一者舉數之由故曰五帝本}\左小列{紀第一禮云動則左史書之言則右史書之左}}
\缩进[3]\双列{\右小列{陽故記動右隂故記言言為尚書事为}\左小列{春秋按春秋时置左右史故云史记}}
黄帝者\双列{\右小列{集解徐廣曰號有熊索隠按有土徳之瑞土色}\左小列{黄故稱黄帝猶神農火徳王而稱炎帝然也此}}
\换页
\双列{\右小列{以黄帝為五帝之首蓋依大戴禮五帝徳又譙周宋均}\左小列{亦以為然而孔安國皇甫謐帝王代紀及孫氏註系本}}
\双列{\右小列{並以伏犧神農黄帝為三皇少昊髙陽髙辛唐虞為五}\左小列{帝註號有熊者以其本是有熊國君之子故也都軒轅}}
\双列{\右小列{之丘因以為名又以為號又據左傳亦號帝鴻氏也正}\左小列{義輿地志云涿鹿本名彭城黄帝初都遷有熊也按黄}}
\双列{\右小列{帝有熊國君乃少典國君之次子號曰有熊氏又曰縉}\左小列{雲氏又曰帝鴻氏亦曰帝軒氏母曰附寶之祁野見大}}
\双列{\右小列{電繞北斗樞星感而懐孕二十四月而生黄帝於壽丘}\左小列{夀丘在魯東門之北今在兖州曲阜縣東北六里生日}}
\双列{\右小列{角龍顔有景雲之瑞以土徳王故曰}\左小列{黄帝封泰山禪亭亭亭亭在牟隂}}少典之子\双列{\右小列{集解譙}\左小列{周曰有}}
\双列{\右小列{熊國君少典之子也皇甫謐曰有熊今河南新鄭是也}\左小列{索隠少典者諸侯國號非人名也又按國語云少典娶}}
\双列{\右小列{有蟜氏女而生炎帝然則炎帝亦少典之子炎黄二帝}\左小列{雖則承帝王代紀中間凡隔八帝五百餘年若以少典}}
\双列{\右小列{是其父名豈黄帝經五百餘年而始代炎帝後為天子}\左小列{乎何其年之长也又按秦本紀云顓頊氏之裔孫曰女}}
\双列{\右小列{脩吞玄鳥之卵而生大业大业娶少典氏而生栢翳明}\左小列{少典是国号非人名也黄帝者少典氏后代之子孙贾}}
\双列{\右小列{逵亦以左传高阳氏有才子八人亦谓其后代子孙而}\左小列{称为子是也谯周字允南蜀人魏散骑常侍征不拜此}}
\双列{\右小列{注所引者是其人所着古史考之说也皇甫谧字士安}\左小列{晋人号玄晏先生今所引者是其所作帝王世纪也}}
姓公孫名曰軒轅\双列{\右小列{索隠按皇甫謐云黄帝生於壽丘長}\左小列{於姬水因以為姓居軒轅之丘因以}}
\双列{\右小列{為名又以為號是本姓公}\左小列{孫長居姬水因改姓姬}}生而神靈弱而能言\双列{\右小列{索隠弱}\左小列{謂幼弱}}
\双列{\右小列{时也盖未合能言之时而黄帝即言所以为神异也潘}\左小列{岳有哀弱子篇其子未七旬曰弱正义言神异也易曰}}
\双列{\右小列{阴阳不测之谓神书曰人惟}\左小列{万物之灵故谓之神灵也}}幼而徇齊\双列{\右小列{集解徐廣曰墨}\左小列{子曰年踰十五}}

\end{数字化内容}
\end{document}
