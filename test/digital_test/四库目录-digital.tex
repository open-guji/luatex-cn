\documentclass[SikuWenyuanMulu]{guji-digital}

\无标点模式
\setmainfont{TW-Kai}

\title{欽定四庫全書簡明目錄}
\chapter{經部 易類\\卷一}

\begin{document}
\begin{数字化内容}
\印章[page=1,opacity=0.7,color=black,xshift=5.3cm,yshift=6.7cm,width=12.9cm]{文渊阁宝印.png}%
欽定四庫全書簡明目錄卷一
\缩进[1] 經部一
\缩进[2] 易類
子夏易傳十一卷
\缩进[2]\双列{\右小列{舊本題卜子夏撰實後人輾轉依托非其原書然}\左小列{唐宋以來流傳已久今仍錄冠易類之首凡托名}}
\缩进[2]\双列{\右小列{之書仍從其所托之時代漢書藝文志例也}\左小列[indent=4]{謹按唐徐堅初學記以太宗御制升列歷代}}
\缩进[4]\双列{\右小列{之前蓋尊尊之大義焦竑國史經籍志朱彞}\左小列{尊經義考並踵前規臣等編摩四庫初亦恭}}
\缩进[4]\双列{\右小列{錄}\左小列[indent=-1]{御定易經通注}}
\缩进[-1]\双列{\右小列{御纂周易折中}\左小列[indent=0]{御纂周易述義弁冕諸經仰蒙}}
\缩进[0]\双列{\右小列{指示命冠於}\左小列[indent=2]{國朝著述之首俾尊卑有序而時代不淆}}
\缩进[0]\双列{\右小列{聖度謙沖酌中立憲實為千古之大公謹恪遵}\左小列{彞訓仍託始於子夏易傳並發凡於此著四庫之通例}}
\缩进[4]\双列{\右小列{焉}\左小列{}}
周易鄭康成注一卷
\缩进[2]\双列{\右小列{漢鄭玄撰原本散佚此本乃宋末王應麟采諸書}\左小列{所引裒合而成}}
\缩进[4]\双列{\右小列{前代佚書而後人重編者如有所竄改則從}\左小列{重編之時代如全輯舊文者則仍從原書之}}
\缩进[4]\双列{\右小列{時代故此書雖宋人所輯而列於漢代之中}\左小列{後皆仿此}}

\end{数字化内容}
\end{document}
