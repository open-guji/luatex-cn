\documentclass{cn-vbook}
\setmainfont{TW-Kai}

\begin{document}

% ============================================================================
% 第一页:标点符号全面测试(大陆模式 punct-style=mainland)
% ============================================================================
\begin{正文}

% 基本标点:逗号、顿号(半角挤压、偏右)
天地玄黄,宇宙洪荒。日月盈昃,辰宿列张。

% 顿号
金、木、水、火、土,五行也。

% 冒号、分号
子曰:学而时习之;不亦说乎?有朋自远方来!

% 问号、叹号
何谓仁?何谓义?仁者爱人!义者正也!

% 引号(「」『』竖排形式)
子曰:「学而时习之,不亦说乎?」又曰:「温故而知新。」

% 嵌套引号
他说:「我读过『论语』。」她说:「我也读过。」

% 括号
(括号内容)接续文字。〈角括号〉接续。《书名号》接续。【方括号】接续。

% 破折号、省略号
注释——重要说明。省略……继续文字。

% 连续标点
「好。」「好。」连续引号测试。

% 标点在列首/列尾测试
一二三四五六七八九十一二三四五六七八九十,逗号避头测试

一二三四五六七八九十一二三四五六七八九十。句号避头测试

\end{正文}

\newpage

% ============================================================================
% 第二页:段落与缩进
% ============================================================================
\begin{正文}

% 默认正文
天地玄黄,宇宙洪荒。日月盈昃,辰宿列张。

% 段落环境(缩进2格)
\begin{段落}[indent=2]
寒来暑往,秋收冬藏。闰余成岁,律吕调阳。
\end{段落}

% 抬头系列(在段落中使用)
\begin{段落}[indent=2]
缩进两格\平抬 云腾致雨,露结为霜。
缩进两格\单抬 金生丽水,玉出昆冈。
缩进两格\双抬 剑号巨阙,珠称夜光。
\end{段落}

% 挪抬
果珍李柰,\挪抬 菜重芥姜。

% 换行(强制换列)
海咸河淡,\换行 鳞潜羽翔。

% 设置缩进
\设置缩进{3}龙师火帝,鸟官人皇。

% 空格
始制\空格 文字,乃服\空格[2]衣裳。

\end{正文}

\newpage

% ============================================================================
% 第三页:行/列布局与文本框
% ============================================================================
\begin{正文}

% Column 不同对齐方式
\行{默认顶部对齐}
\行[align=center]{居中对齐}
\行[align=bottom]{底部对齐}
\行[align=stretch]{拉伸对齐测试文字}

% 带颜色的行
\行[font-color=red]{红色文字}

% Style 样式命令测试
一二三\Style[font-color=red]{红色样式}四五
一二三\样式[font-size=8pt]{小字样式}四五

% 末行
一二三四五
\末行{末列内容}

% TextBox 文本框
\文本框{文本框内容}

% FillTextBox 填充文本框
\填充文本框{填充框}

% 反白
\反白{反白效果}

% 八角框
\八角框{八角框}

% 带圈
\带圈{圈}

\end{正文}

\newpage

% ============================================================================
% 第四页:侧批、脚注、装饰、线标记
% ============================================================================
\begin{正文}
\脚注设置{mode=endnote, number-style=lujiao, indent=1em}

% 侧批
日月盈昃\侧批{日月运行},辰宿列张。

% 脚注
寒来暑往\脚注{四时更替之意。},秋收冬藏\脚注{农事规律。}。

\输出脚注

% 装饰:着重号
\着重号{金生丽水},玉出昆冈。

% 装饰:自定义
\装饰[char=、, color=blue]{剑号巨阙},珠称夜光。

% 改字
天\改{地}改字功能测试。

% 专名号
\专名号{司马迁}撰\书名号{史记}。

% 书名号
\书名号{论语}为儒家经典。

\end{正文}

\end{document}
