\documentclass{ltc-guji}
\关闭分页

\begin{document}
\begin{正文}

% 基本用法:默认圈点
\decorate{天地}默认圈点

% 自定义装饰字符
\decorate[char=、]{宇宙}顿号装饰

% 颜色测试
\decorate[char=。, color=blue]{日月}蓝色圈点

\decorate[char=●, color=green, scale=0.4]{辰宿}绿色实心点

% 位置偏移测试
\decorate[char=。, xshift=-0.3em]{寒来}右偏移

\decorate[char=。, yshift=0.3em]{秋收}下偏移

% 缩放测试
\decorate[char=。, scale=0.5]{闰余}缩小圈点

\decorate[char=●, scale=0.6, color=red]{律吕}红色实心点

% 混合使用
云腾\decorate[char=。]{致雨}露结\decorate[char=、]{为霜}混合装饰

% 改字功能
天\改{地}改字测试

% EmphasisMark 着重号功能
\EmphasisMark{金生}着重号默认

\圈点[color=blue]{丽水}蓝色圈点

\着重号[color=green]{玉出}绿色着重号

% ============================================================================
% 专名号/书名号 (LineMark) - PDF 画线实现
% ============================================================================

% 专名号(直线)- 基本用法
\专名号{司馬遷}默认专名号

% 专名号 - 颜色
\专名号[color=blue]{班固}蓝色专名号

% 书名号(波浪线)- 基本用法
\书名号{史記}默认书名号

% 书名号 - 颜色
\书名号[color=blue]{漢書}蓝色书名号

% 连续词语断开测试:两个书名号之间应有小间距
\书名号{史記}\书名号{漢書}连续书名

% 连续词语断开测试:两个专名号之间
\专名号{司馬遷}\专名号{班固}连续专名

% 混合测试:专名号和书名号相邻
\专名号{司馬遷}撰\书名号{史記}混合

% 波浪线振幅测试(标准样式 - 默认)
\书名号[amplitude=small]{小振幅}小

\书名号[amplitude=medium]{中振幅}中

\书名号[amplitude=large]{大振幅}大

% 波浪线样式测试:草书风格
\书名号[style=cursive]{草书波浪}草书

\书名号[style=cursive, amplitude=large]{大草书}大草

% 英文别名兼容性
\Underline{下划线}兼容

\WavyUnderline{波浪线}兼容

% ============================================================================
% 特殊环境中的专名号/书名号
% ============================================================================

% 夹注中的专名号(默认 outward 对齐)
前文\夹注{\专名号{司馬遷}著\书名号{史記}}夹注内

% 夹注中的书名号(默认 outward 对齐)
前文\夹注{\书名号{漢書}由\专名号{班固}撰}夹注内

% 夹注 inward 对齐
前文\夹注[align=inward]{\专名号{司馬遷}著\书名号{史記}}向内

% 夹注 center 对齐
前文\夹注[align=center]{\专名号{司馬遷}著\书名号{史記}}居中

% 夹注 left 对齐
前文\夹注[align=left]{\专名号{司馬遷}著\书名号{史記}}居左

% 侧批中的专名号
\侧批{\专名号{杜甫}}诗圣杜甫

% 文本框中的专名号
\文本框{\专名号{李白}\书名号{蜀道難}}

\end{正文}
\end{document}
