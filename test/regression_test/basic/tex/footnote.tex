\documentclass{ltc-cn-vbook}
\setmainfont{TW-Kai}
\开启调试

\begin{document}
\begin{正文}
\脚注设置{mode=endnote, number-style=lujiao, indent=1em}

天地玄黃\脚注{「玄」,本作「元」,避清聖祖諱改。},宇宙洪荒\脚注{「洪荒」,謂天地初開之際。}。
日月盈昃,辰宿列張。

\输出脚注

寒來暑往,秋收冬藏。閏餘成歲\脚注{「閏餘」,指閏月。},律呂調陽。

\输出脚注

\end{正文}

\newpage
\脚注设置{mode=page, number-style=lujiao, indent=1em}
\begin{正文}

天地玄黃\脚注{「玄」,本作「元」,避清聖祖諱改。},宇宙洪荒\脚注{「洪荒」,謂天地初開之際。}。
日月盈昃,辰宿列張。
寒來暑往,秋收冬藏。閏餘成歲\脚注{「閏餘」,指閏月。},律呂調陽。

雲騰致雨,露結為霜。金生麗水,玉出崑岡。
劍號巨闕,珠稱夜光。果珍李柰,菜重芥薑。
海鹹河淡,鱗潛羽翔。龍師火帝,鳥官人皇。

始制文字,乃服衣裳。推位讓國,有虞陶唐。
弔民伐罪,周發殷湯。坐朝問道,垂拱平章。
\end{正文}
\end{document}
