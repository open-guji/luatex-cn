\documentclass{tw-vbook}
\setmainfont{TW-Kai}
\禁用分页裁剪

\begin{document}

% Test 1: Basic wrapping — 12pt in 10-row column, triggers multi-column
\Content[
    layout-mode = natural,
    font-size = 12pt,
    n-column = 8,
    n-char-per-col = 10,
]{
    這是自然排版模式的測試文本。每個字的格高由字號決定,而不是固定的網格高度。自動換列功能應該正常工作。標點符號只佔半格,這樣可以節省空間。
}

\newpage

% Test 2: Dense punctuation — half-cell sizing, tight packing
\Content[
    layout-mode = natural,
    font-size = 12pt,
    n-column = 8,
    n-char-per-col = 10,
]{
    子曰:「學而時習之,不亦說乎?」有朋自遠方來,不亦樂乎?人不知而不慍,不亦君子乎?
}

\newpage

% Test 3: Long text — sustained wrapping across many columns
\Content[
    layout-mode = natural,
    font-size = 12pt,
    n-column = 8,
    n-char-per-col = 10,
]{
    天地玄黃,宇宙洪荒。日月盈昃,辰宿列張。寒來暑往,秋收冬藏。閏餘成歲,律呂調陽。雲騰致雨,露結為霜。金生麗水,玉出崑岡。劍號巨闕,珠稱夜光。果珍李柰,菜重芥薑。海鹹河淡,鱗潛羽翔。龍師火帝,鳥官人皇。始制文字,乃服衣裳。推位讓國,有虞陶唐。弔民伐罪,周發殷湯。坐朝問道,垂拱平章。
}

\newpage

% Test 4: Large font — fewer chars per column, more frequent wrapping
\Content[
    layout-mode = natural,
    font-size = 20pt,
    n-column = 8,
    n-char-per-col = 10,
]{
    大字排版測試。每個字佔更多空間,換列更頻繁。標點「如此」佔半格。
}

\end{document}
