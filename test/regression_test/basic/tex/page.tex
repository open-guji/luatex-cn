\documentclass{guji}
\setmainfont{TW-Kai}
\显示坐标

\begin{document}

\begin{封面}[底色={245,222,179}]
  % 永樂大典 - 右上角竖排标题框(双层边框)
  \TextBox[
    floating=true,
    x=15.3cm,
    y=0.3cm,
    grid-height=56pt,
    grid-width=92pt,
    border-shape=rect,
    border-width=2pt,
    border-margin=4pt,
    outer-border=true,
    outer-border-thickness=6pt,
    outer-border-sep=4pt,
    font-size=52pt,
    height=24cm
  ]{欽定四庫全書\夹注[align=center,font-size=20pt]{\样式[grid-height=24pt]{經部\空格[4] 簡明目錄卷一}}}
\end{封面}


% 测试书名页环境(使用 Column 的 width 参数)
\begin{书名页}
  \行[width=5cm, align=top, font-size=48pt, grid-height=54pt]{光緒九年仲春}
  \行[width=6cm, align=center, font-size=80pt, grid-height=90pt]{童蒙必讀書}
  \行[width=5cm, align=bottom, font-size=48pt, grid-height=54pt]{武昌書局校刊}
\end{书名页}

% 测试书名页环境(文本框自由排版)
\begin{空白页}
  \文本框[
    floating=true,
    x=5cm,
    y=10cm,
    font-size=28pt,
    grid-height=28pt,
    height=8]{檢討\臣 翁樹培覆勘}
  \文本框[
    floating=true,
    x=13cm,
    y=15cm,
    font-size=28pt,
    grid-height=28pt,
    height=11]{協勘官候補洗馬\臣 劉權之}
  \文本框[
    floating=true,
    x=14.5cm,
    y=16.8cm,
    font-size=28pt,
    grid-height=28pt,
    height=10]{\样式[grid-height=36pt]{謄錄舉人}\臣 李學沅}
\end{空白页}

\begin{BodyText}
默认页面尺寸
\end{BodyText}

\begin{Page}[paper-width=5cm, paper-height=10cm, page-split-enabled=false]
第二页
自定义页面:宽五厘米、高十厘米
\end{Page}

\begin{BodyText}
第三页
恢复默认尺寸
\end{BodyText}

\begin{单页}
\begin{BodyText}
\文本框{第四页}
\文本框{单页模式:自动计算宽度为原来的一半}
\end{BodyText}
\end{单页}

\begin{BodyText}
第五页
恢复默认尺寸
\end{BodyText}

\end{document}
