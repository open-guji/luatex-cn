\documentclass{tw-vbook}
\setmainfont{TW-Kai}

\begin{document}

% ============================================================================
% 第一頁:標點符號全面測試(台灣模式 punct-style=taiwan)
% 標點應全角居中,不進行擠壓
% ============================================================================
\begin{正文}

% 基本標點:逗號、頓號(全角居中)
天地玄黃,宇宙洪荒。日月盈昃,辰宿列張。

% 頓號
金、木、水、火、土,五行也。

% 冒號、分號
子曰:學而時習之;不亦說乎?有朋自遠方來!

% 問號、嘆號
何謂仁?何謂義?仁者愛人!義者正也!

% 引號(「」『』豎排形式)
子曰:「學而時習之,不亦說乎?」又曰:「溫故而知新。」

% 巢狀引號
他說:「我讀過『論語』。」她說:「我也讀過。」

% 括號
(括號內容)接續文字。〈角括號〉接續。《書名號》接續。【方括號】接續。

% 破折號、省略號
注釋——重要說明。省略……繼續文字。

% 連續標點
「好。」「好。」連續引號測試。

% 標點在列首/列尾測試
一二三四五六七八九十一二三四五六七八九十,逗號避頭測試

一二三四五六七八九十一二三四五六七八九十。句號避頭測試

\end{正文}

\newpage

% ============================================================================
% 第二頁:段落與縮排
% ============================================================================
\begin{正文}

% 預設正文
天地玄黃,宇宙洪荒。日月盈昃,辰宿列張。

% 段落環境(縮排2格)
\begin{段落}[indent=2]
寒來暑往,秋收冬藏。閏餘成歲,律呂調陽。
\end{段落}

% 抬頭系列(在段落中使用)
\begin{段落}[indent=2]
縮排兩格\平抬 雲騰致雨,露結為霜。
縮排兩格\单抬 金生麗水,玉出崑岡。
縮排兩格\双抬 劍號巨闕,珠稱夜光。
\end{段落}

% 挪抬
果珍李柰,\挪抬 菜重芥薑。

% 換行(強制換列)
海鹹河淡,\换行 鱗潛羽翔。

% 設置縮排
\设置缩进{3}龍師火帝,鳥官人皇。

% 空格
始制\空格 文字,乃服\空格[2]衣裳。

\end{正文}

\newpage

% ============================================================================
% 第三頁:行/列佈局與文字框
% ============================================================================
\begin{正文}

% Column 不同對齊方式
\行{預設頂部對齊}
\行[align=center]{居中對齊}
\行[align=bottom]{底部對齊}
\行[align=stretch]{拉伸對齊測試文字}

% 帶顏色的行
\行[font-color=red]{紅色文字}

% 末行
一二三四五
\末行{末列內容}

% TextBox 文字框
\文本框{文字框內容}

% FillTextBox 填充文字框
\填充文本框{填充框}

% 反白
\反白{反白效果}

% 八角框
\八角框{八角框}

% 帶圈
\带圈{圈}

\end{正文}

\newpage

% ============================================================================
% 第四頁:側批、腳注、裝飾、線標記
% ============================================================================
\begin{正文}
\脚注设置{mode=endnote, number-style=lujiao, indent=1em}

% 腳注
寒來暑往\脚注{四時更替之意。},秋收冬藏\脚注{農事規律。}。

\输出脚注

% 裝飾:著重號
\着重号{金生麗水},玉出崑岡。

% 裝飾:自訂
\装饰[char=、, color=blue]{劍號巨闕},珠稱夜光。

% 改字
天\改{地}改字功能測試。

% 專名號
\专名号{司馬遷}撰\书名号{史記}。

% 書名號
\书名号{論語}為儒家經典。

% Style 樣式命令測試
一二三\Style[font-color=red]{紅色樣式}四五
一二三\样式[font-size=8pt]{小字樣式}四五

\end{正文}

\end{document}
