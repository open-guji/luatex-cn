\documentclass{guji}
\setmainfont{TW-Kai}
\禁用分页裁剪
\无标点模式

\NewDocumentCommand{\注}{ +m }{%
  \begin{段落}[indent=2]
    \夹注[auto-balance=false]{#1}%
  \end{段落}%
}
\NewDocumentCommand{\國朝}{}{\相对抬头[1]{國朝}}

\begin{document}
\begin{正文}
% 正好21字满行
《周易象辭》二十一卷,附《尋門餘論》二卷,《圖書辨惑》一卷

\注{\國朝 黃宗炎撰。宗炎力闢陳摶之學,故所解惟主義理,然根據經典,不涉空談。「尋門餘論」兼排釋氏,未免蔓衍於《易》外,而其他持論多醇正。「圖書辨惑」論《先天圖》,與陳應潤所言合;論《太極圖》,與朱彞尊、毛奇齡所考合。亦皆明確也。}

\end{正文}
\end{document}
