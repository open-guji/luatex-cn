\documentclass{ltc-guji}

\usepackage{enumitem}
\usepackage{tikz}
\setmainfont{TW-Kai}
\禁用分页裁剪

\begin{document}
\begin{正文}
正文测试抬头命令

\begin{段落}[indent=2]
缩进两格的段落\平抬 平抬顶格书写(御定易经通注)
继续内容
\end{段落}

下一段

\begin{段落}[indent=2]
缩进两格的段落\单抬 单抬高出一格(皇帝御制)
继续内容
\end{段落}

下一段

\begin{段落}[indent=2]
缩进两格的段落\双抬 双抬高出两格(皇帝圣天子)
继续内容
\end{段落}

下一段

\begin{段落}[indent=2]
缩进两格的段落\三抬 三抬高出三格(列祖列宗)
继续内容
\end{段落}

下一段

\begin{段落}[indent=2]
缩进两格的段落\抬头[0] 抬头零等同平抬
继续内容
\end{段落}

挪抬测试

\begin{段落}[indent=2]
缩进两格然后\挪抬 空一格后继续\\
缩进两格然后\挪抬[2] 空两格后继续\\
缩进两格然后\空抬 空抬一格后继续
\end{段落}

相对抬头测试

\begin{段落}[indent=2]
缩进两格\相对抬头 上移一格变缩进一格
\end{段落}

\begin{段落}[indent=2]
缩进两格\相对抬头[2] 上移两格变顶格
\end{段落}

\begin{段落}[indent=2]
缩进两格\相对抬头[3] 上移三格伸入天头一格
\end{段落}

连续抬头不同高度测试

\begin{段落}[indent=2]
缩进两格的段落\双抬 双抬高出两格\单抬 单抬高出一格
继续缩进内容
\end{段落}

夹注中的抬头测试

\begin{段落}[indent=2]
\夹注[auto-balance=false]{缩进两格的夹注内容。\单抬 御定易经通注}
\end{段落}

\begin{段落}[indent=2]
夹注内挪抬测试\夹注{正常内容\挪抬 空一格后继续内容结束}
正文继续
\end{段落}

\end{正文}
\end{document}
