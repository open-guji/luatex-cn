\documentclass[四库全书]{ltc-guji}
% 可选:指定字体
\setmainfont{ZhongHuaSongPlane00}
\句读模式
\禁用分页裁剪

\title{钦定四库全书}
\chapter{史记\\卷一}

\begin{document}
\begin{正文}

這是古籍竖排的示例文本。

\专名号[offset=0.6em]{黄帝}者,\夹注{徐廣曰號有熊。}\专名号[offset=0.6em]{少典}之子,姓\专名号[offset=0.6em]{公孫},名曰\专名号[offset=0.6em]{軒轅}。生而神靈,弱而能言,幼而徇齊,長而敦敏,成而聰明。

\专名号[offset=0.6em]{軒轅}之時,\专名号[offset=0.6em]{神農氏}世衰,諸侯相侵伐,暴虐百姓,而\专名号[offset=0.6em]{神農氏}弗能征。於是\专名号[offset=0.6em]{軒轅}乃習用干戈以征不享,諸侯\书名号[offset=0.4em]{咸來賓從}。而\专名号[offset=0.6em]{蚩尤}最爲暴,莫能伐。

\书名号[offset=0.4em]{史記}原名\书名号[offset=0.4em]{太史公書},\专名号[offset=0.6em]{司馬遷}撰。\专名号[offset=0.6em]{司馬遷}字\专名号[offset=0.6em]{子長},\专名号[offset=0.6em]{漢}左馮翊\专名号[offset=0.6em]{夏陽}(今\专名号[offset=0.6em]{陝西}\专名号[offset=0.6em]{韓城縣})人,生於漢景帝中元五年(公元前一四五)或者更後一些。他的父親\专名号[offset=0.6em]{司馬談},熟悉史事,懂天文地理。\专名号[offset=0.6em]{漢武帝}建元(公元前一四〇——一三五)初年,做了太史令(\书名号[offset=0.4em]{史記}中稱爲\专名号[offset=0.6em]{太史公})。他早就有意論載「天下之史文」,但始終沒有如願。

\end{正文}
\end{document}