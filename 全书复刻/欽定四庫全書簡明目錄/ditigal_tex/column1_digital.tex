\documentclass[四库全书文渊阁简明目录]{guji-digital}

\无标点模式
\setmainfont{TW-Kai}

\title{欽定四庫全書簡明目錄}

\begin{document}


\begin{封面}[底色={245,222,179}]
  % 永樂大典 - 右上角竖排标题框(双层边框)
  \TextBox[
    floating=true,
    x=16.5cm,
    y=1.5cm,
    grid-height=55pt,
    grid-width=55pt,
    border-shape=rect,
    border-width=0.8pt,
    border-margin=2pt,
    outer-border=true,
    outer-border-thickness=1.2pt,
    outer-border-sep=4pt,
    font-size=50pt, height=10]{欽定四庫全書\空格 經部}
\end{封面}

\begin{书名页}
  \行[width=4cm, align=top, font-size=36pt, grid-height=36pt]{欽定四庫全書 \夹注{經部\空格[4] 簡明目錄卷一}}
  \行[width=3cm, align=bottom, font-size=24pt, grid-height=24pt]{檢討臣翁樹培覆勘}
  \行[width=3cm, align=bottom, font-size=24pt, grid-height=24pt]{協勘官候補洗馬臣劉權之}
  \行[width=3cm, align=bottom, font-size=24pt, grid-height=24pt]{謄錄舉人臣李學沅}
\end{书名页}

\chapter{聖諭}
\begin{数字化内容}
欽定四庫全書簡明目錄
聖諭
\缩进[2] 乾隆三十九年七月二十五日大學士于敏中等
\缩进[2] 奉
諭旨四庫全書處總目於經史子集內分晰應刻應抄
\缩进[1] 及應存書名三項各條下俱經撰有提要將一書原
\缩进[1] 委撮舉大凡并詳著書人世次爵里可以一覽了然
\缩进[1] 較之崇文總目範羅既廣體例加詳自應如此辦理
\缩进[1] 第此次各省搜訪書籍有多至百種以上至六七百
\缩进[1] 種者如浙江范懋柱等家其裒集收藏深可嘉尚前
\缩进[1] 已降旨分別頒賞古今圖書集成及初印佩文韻府
\缩进[1] 并擇其書尤雅者製詩親題卷端俾其子孫世守以
\缩进[1] 為稽古藏書者勸今進到之書於攥輯後仍須發還
\缩进[1] 本家而所撰總目若不載明係何人所藏則閱者不
\缩进[1] 能知其書所自來亦無以彰各家珍弃資益之善著
\缩进[1] 通查各省進到之書其一人而收藏百種以上者可
\缩进[1] 稱為藏書之家即應將其姓名附載于各書提要末
\缩进[1] 其在百種以下者亦應將由其省督撫主人採訪所
\缩进[1] 得附載於後其官板刊刻及各處陳設庫貯者俱載
\缩进[1] 內府所藏使其眉目分明更為詳細至現辦四庫全
\缩进[1] 書總目提要多至萬餘種卷帙甚繁將來抄刻成書
\缩进[1] 總閱已頗為不易自應於提要之外別刊簡明書目
\缩进[1] 一編祗載其書若干卷註其朝代人撰則篇幅不繁
\缩进[1] 而檢查較易俾學者由書目而尋提要由提要而得
\缩进[1] 全書嘉與海內之士考訂源流用昭我朝文治之盛
\缩进[1] 著四庫全書處總裁等遵照悉心妥辦並著通諭知
\缩进[1] 之欽此
\end{数字化内容}
\newpage
\chapter{目錄}
\begin{数字化内容}
欽定四庫全書簡明目錄
\缩进[1] 卷一
\缩进[2] 經部一
\缩进[3] 易類
\缩进[1] 卷二
\缩进[2] 經部二
\缩进[3] 書類
\缩进[2] 經部三
\缩进[3] 詩類
\缩进[2] 經部四
\缩进[3] 禮類
\缩进[1] 卷三
\缩进[2] 經部五
\缩进[3] 春秋類
\缩进[2] 經部六
\缩进[3] 孝經類
\缩进[2] 經部七
\缩进[3] 五經總義類
\缩进[1] 卷四
\缩进[2] 經部八
\缩进[3] 四書類
\缩进[2] 經部九
\缩进[3] 樂類
\缩进[2] 經部十
\缩进[3] 小學類
\缩进[1] 卷五
\缩进[2] 史部一
\缩进[3] 正史類
\缩进[2] 史部二
\缩进[3] 編年類
\缩进[2] 史部三
\缩进[3] 紀事本末類
\缩进[2] 史部四
\缩进[3] 別史類
\缩进[2] 史部五
\缩进[3] 雜史類
\缩进[1] 卷六
\缩进[2] 史部六
\缩进[3] 詔令奏議類
\缩进[2] 史部七
\缩进[3] 傳記類
\缩进[2] 史部八
\缩进[3] 史鈔類
\缩进[2] 史部九
\缩进[3] 載記類
\缩进[1] 卷七
\缩进[2] 史部十
\缩进[3] 時令類
\缩进[2] 史部十一
\缩进[3] 地理類
\缩进[1] 卷八
\缩进[2] 史部十二
\缩进[3] 職官類
\缩进[2] 史部十三
\缩进[3] 政書類
\缩进[2] 史部十四
\缩进[3] 目錄類
\缩进[2] 史部十五
\缩进[3] 史評類
\缩进[1] 卷九
\缩进[2] 子部一
\缩进[3] 儒家類
\缩进[2] 子部二
\缩进[3] 兵家類
\缩进[1] 卷十
\缩进[2] 子部三
\缩进[3] 法家類
\缩进[2] 子部四
\缩进[3] 農家類
\缩进[2] 子部五
\缩进[3] 醫家類
\缩进[1] 卷十一
\缩进[2] 子部六
\缩进[3] 天文算法類
\缩进[2] 子部七
\缩进[3] 術數類
\缩进[1] 卷十二
\缩进[2] 子部八
\缩进[3] 藝術類
\缩进[2] 子部九
\缩进[3] 譜錄類
\缩进[1] 卷十三
\缩进[2] 子部十
\缩进[3] 雜家類
\缩进[1] 卷十四
\缩进[2] 子部十一
\缩进[3] 類書類
\缩进[2] 子部十二
\缩进[3] 小說家類
\缩进[2] 子部十三
\缩进[3] 釋家類
\缩进[2] 子部十四
\缩进[3] 道家類
\缩进[1] 卷十五
\缩进[2] 集部一
\缩进[3] 楚詞類
\缩进[2] 集部二
\缩进[3] 別集類一漢至五代
\缩进[2] 集部三
\缩进[3] 別集類二北宋建隆至靖康
\缩进[1] 卷十六
\缩进[2] 集部四
\缩进[3] 別集類三南宋建炎至德祐
\缩进[1] 卷十七
\缩进[2] 集部五
\缩进[3] 別集類四金至元
\缩进[1] 卷十八
\缩进[2] 集部六
\缩进[3] 別集類五明洪武至崇禎
\缩进[2] 集部七
\缩进[3] 別集類六\挪抬 國朝 
\缩进[1] 卷十九
\缩进[2] 集部八
\缩进[3] 總集類
\缩进[1] 卷二十
\缩进[2] 集部九
\缩进[3] 詩文評類
\缩进[2] 集部十
\缩进[3] 詞曲類
\end{数字化内容}
\newpage
\chapter{經部 易類\\卷一}
\begin{数字化内容}
欽定四庫全書簡明目錄卷一
\印章[page=12,opacity=0.7,color=black,xshift=5.3cm,yshift=6.7cm,width=12.9cm]{文渊阁宝印.png}%
\缩进[1] 經部一
\缩进[2] 易類
子夏易傳十一卷
\缩进[2]\双列{\右小列{舊本題卜子夏撰實後人輾轉依托非其原書然}\左小列{唐宋以來流傳已久今仍錄冠易類之首凡托名}}
\缩进[2]\双列{\右小列{之書仍從其所托之時代漢書藝文志例也}\左小列[indent=4]{謹按唐徐堅初學記以太宗御制升列歷代}}
\缩进[4]\双列{\右小列{之前蓋尊尊之大義焦竑國史經籍志朱彞}\左小列{尊經義考並踵前規臣等編摩四庫初亦恭}}
\缩进[4]\双列{\右小列{錄}\左小列[indent=-1]{御定易經通注}}
\缩进[-1]\双列{\右小列{御纂周易折中}\左小列[indent=0]{御纂周易述義弁冕諸經仰蒙}}
\缩进[0]\双列{\右小列{指示命冠於}\左小列[indent=1]{國朝著述之首俾尊卑有序而時代不淆}}
\缩进[0]\双列{\右小列{聖度謙沖酌中立憲實為千古之大公謹恪遵}\左小列{彞訓仍託始於子夏易傳並發凡於此著四庫之通例}}
\缩进[4]\双列{\右小列{焉}\左小列{}}
周易鄭康成注一卷
\缩进[2]\双列{\右小列{漢鄭玄撰原本散佚此本乃宋末王應麟采諸書}\左小列{所引裒合而成}}
\缩进[4]\双列{\右小列{前代佚書而後人重編者如有所竄改則從}\左小列{重編之時代如全輯舊文者則仍從原書之}}
\缩进[4]\双列{\右小列{時代故此書雖宋人所輯而列於漢代之中}\左小列{後皆仿此}}
新本鄭氏周易三卷
\缩进[2]\双列{\右小列{漢鄭玄撰}\左小列{國朝惠棟編因王應麟之本採摭未備又不註其}}
\缩进[2]\双列{\右小列{所出因重為補正凡增入九十二條又據鄭氏周}\左小列{禮禮記註作十二月爻辰及爻辰直二十八宿圖}}
\缩进[2]\双列{\右小列{以闡明漢學}\左小列{}}
陸氏易解一卷
\缩进[2]\双列{\右小列{吳陸績撰原本散佚明姚士粦採陸氏經典釋文}\左小列{李氏周易集解及績京氏易傳註輯為此本凡一}}
\缩进[2]\双列{\右小列{百五十條}\左小列{}}
周易註十卷
\缩进[2]\双列{\右小列{魏王弼註其繫辭以下則韓康伯註也漢氏易學}\左小列{皆明象數至弼始黜象數而言義理足以糾讖緯}}
\缩进[2]\双列{\右小列{之失而語涉老莊亦開後來玄虛之漸}\左小列{}}
周易正義十卷
\缩进[2]\双列{\右小列{唐孔穎達撰穎達諸經正義皆元元本本引據詳}\左小列{明惟周易罕徵典籍蓋所疏者王韓之註而王韓}}
\缩进[2]\双列{\右小列{皆掃棄舊聞自標新解故不能以漢儒古義與之}\左小列{證明非其考訂之疏也}}
周易集解十七卷
\缩进[2]\双列{\右小列{唐李鼎祚撰凡採子夏易傳以下三十五家之說}\左小列{鼎祚自序稱刊輔嗣之野文補康成之逸象蓋發}}
\缩进[2]\双列{\右小列{明漢學者也}\左小列{}}
周易口訣義六卷
\缩进[2]\双列{\右小列{唐史徵撰大旨與李鼎祚書相類而與李書互有}\左小列{詳略且多李書所未載世無傳本今從永樂大典}}
\缩进[2]\双列{\右小列{錄出為罕觀之祕笈}\左小列{}}
周易舉正三卷
\缩进[2]\双列{\右小列{舊本題唐郭京撰自序稱得晉王弼韓康伯手寫}\左小列{周易真本刊正今本訛脫一百三十五條朱子本}}
\缩进[2]\双列{\右小列{義亦採用其說然唐書藝文志不著錄至北宋始}\左小列{出晁公武等多疑其依托}}
易數鉤隱圖三卷附遺論九事一卷
\缩进[2]\双列{\右小列{宋劉牧撰其說出於陳摶與邵子先天之學異派}\左小列{同源惟以九數為河圖十數為洛書與邵子異宋}}
\缩进[2]\双列{\右小列{人易數以此書為首其遺論九事皆奇偶陰陽之}\左小列{說先儒之所未言者也}}
周易口義十二卷
\缩进[2]\双列{\右小列{宋倪天隱述其師胡瑗之說故曰口義大旨主闡}\左小列{明義理程子之易源從此出}}
溫公易說六卷
\缩进[2]\双列{\右小列{宋司馬光撰其書宋代有兩本皆已散佚此本為}\左小列{永樂大典所載即朱子語錄所謂北方互市之完}}
\缩进[2]\双列{\右小列{本也大旨在闡明人事不主空虛玄妙之說}\左小列{}}
橫渠易說二卷
\缩进[2]\双列{\右小列{宋張載撰文頗簡略蓋無可發揮新義者即不橫}\左小列{生枝節強為敷衍猶有先儒篤實之遺間有引用}}
\缩进[2]\双列{\右小列{老莊語者蓋借以旁證非祖其虛無之談}\左小列{}}
東坡易傳九卷
\缩进[2]\双列{\右小列{宋蘇軾撰其大體近於王弼然弼說惟暢玄風軾}\左小列{說多切人事實不相同朱子作雜學辨嘗摘駁其}}
\缩进[2]\双列{\右小列{中十九條然不害其全書也}\左小列{}}
伊川易傳四卷
\缩进[2]\双列{\右小列{宋程頤撰其門人楊時校正經文用王弼之本惟}\左小列{解上下經彖象及文言亦與弼同大旨黜數而崇}}
\缩进[2]\双列{\右小列{理與邵子各明一義}\左小列{}}
易學辨惑一卷
\缩进[2]\双列{\右小列{宋邵伯溫撰伯溫邵子之子也以同時鄭夬詭稱}\左小列{得邵子之傳所作說易諸書支離破碎多乖經義}}
\缩进[2]\双列{\右小列{因作此書以辨其誣原本久佚今從永樂大典錄}\左小列{出}}
了翁易說一卷
\缩进[2]\双列{\右小列{宋陳瓘撰瓘之學出於邵氏又常質於劉安世故}\左小列{其說理數兼推陳振孫書錄解題頗病其詞旨深}}
\缩进[2]\双列{\右小列{晦然晁公武讀書志則稱其數之多驗云}\左小列{}}
吳園易解九卷
\缩进[2]\双列{\右小列{宋張根撰不主漢儒象數之說亦不主宋代河洛}\左小列{之學詮釋經文頗為簡切末附泰卦論一篇深著}}
\缩进[2]\双列{\右小列{滿盈之戒蓋作於徽宗之世有為而發也}\左小列{}}
周易新講義十卷
\缩进[2]\双列{\右小列{宋耿南仲撰南仲當欽宗之時力主割地為史傳}\左小列{所譏然是書因象詮理隨事示戒乃頗有可取自}}
\缩进[2]\双列{\右小列{序謂易主於無咎無咎在於善補過而大旨歸於}\左小列{無拂天道}}
紫巖易傳十卷
\缩进[2]\双列{\右小列{宋張浚撰其說發揮易理頗為醇正明白惟末卷}\左小列{雜論以九數為河圖主劉牧之說與朱蔡異然亦}}
\缩进[2]\双列{\右小列{無關宏旨也}\左小列{}}
讀易詳說十卷
\缩进[2]\双列{\右小列{宋李光撰舊本散佚今從永樂大典錄出書中於}\左小列{卦爻之辭皆引証史事以君臣立論或不免有所}}
\缩进[2]\双列{\右小列{牽合然意存法戒究勝空談援古事以證爻象始}\左小列{自鄭玄若全經皆證以史則光書其始也}}
易小傳六卷
\缩进[2]\双列{\右小列{宋沈該撰其書以正體發明爻象之旨以變體擬}\左小列{議變動之意其占則全用左傳所載筮例在南宋}}
\缩进[2]\双列{\右小列{人易說之中為獨存古法}\左小列{}}
漢上易集傳十一卷卦圖三卷叢說一卷
\缩进[2]\双列{\右小列{宋朱震撰其書以數為宗闡陳邵河洛先天之學}\左小列{而兼採漢以來卦變互體伏卦反卦諸說頗為蕪}}
\缩进[2]\双列{\右小列{雜然得失互陳存之亦可資參考}\左小列{}}
周易窺餘十五卷
\缩进[2]\双列{\右小列{宋鄭剛中撰其書以伊川易傳主理漢上易傳主}\左小列{數參取兩家發所未盡故名曰窺餘大旨兼採漢}}
\缩进[2]\双列{\右小列{學而增以新義不甚拘守成說然往往愜當於理}\左小列{原本久佚今從永樂大典錄出}}
易璇璣三卷
\缩进[2]\双列{\右小列{宋吳沆撰自序謂上卷明天理之自然中卷講人}\左小列{事之修下卷備傳疏之失凡論二十七篇其曰璇}}
\缩进[2]\双列{\右小列{璣者取易略例處璇璣以觀大運語也}\左小列{}}
易變體義十二卷
\缩进[2]\双列{\右小列{宋都潔撰其書專明變體即左傳所載諸占某卦}\左小列{之某卦者是也原本久佚今從永樂大典錄出}}
周易經傳集解三十六卷
\缩进[2]\双列{\右小列{宋林慄撰其說每卦必兼言互體約象複卦嘗與}\左小列{朱子論太極兩儀四象八卦不合至於互劾故講}}
\缩进[2]\双列{\右小列{學家最惡其書幾於不傳然易道廣大各明一義}\左小列{不必定執門戶之見也}}
易原八卷
\缩进[2]\双列{\右小列{宋程大昌撰其書推闡數學故謂之易原於京焦}\左小列{卦氣馬鄭爻辰以及邵子張行成諸說皆一一掊}}
\缩进[2]\双列{\右小列{擊務申己說未免失之好辨而根據繫辭於易義}\左小列{亦有所發明非盡鑿空立異舊本久佚今從永樂}}
\缩进[2]\双列{\右小列{大典錄出}\左小列{}}
周易古占法一卷古周易章句外編一卷
\缩进[2]\双列{\右小列{宋程迥撰舊本傳寫混二書為一今考宋史釐正}\左小列{前卷論占法後卷雜說易義及占驗其說用邵子}}
\缩进[2]\双列{\右小列{加一倍法據繫辭說卦發明其義用逆數以尚占}\左小列{知來}}
原本周易本義十二卷
\缩进[2]\双列{\右小列{宋朱熹撰坊刻此書皆改從程傳之次第此本以}\左小列{經為二卷十翼為十卷猶朱子之原本也}}
別本周易本義四卷
\缩进[2]\双列{\右小列{{{{謹案總目此部不存}}}明成矩撰割}\左小列{裂朱子易本義以附程傳之後始元董楷而明永}}
\缩进[2]\双列{\右小列{樂大全因之後場屋專用本義而大全以官本不}\左小列{敢改矩因刊為是本以調停其間相沿日久今亦}}
\缩进[2]\双列{\右小列{姑與原本存焉}\左小列{}}
郭氏傳家易說十一卷
\缩进[2]\双列{\右小列{宋郭雍撰以述其父忠孝兼山易解之旨故名曰}\左小列{傳家自序謂易之為書其道其詞皆由象出未有}}
\缩进[2]\双列{\右小列{忘象而知易者大旨以觀象為主然剖析義理猶}\左小列{守程門之規範蓋其父忠孝即程子之門人也}}
周易義海撮要十二卷
\缩进[2]\双列{\右小列{宋李衡刪定初熙寧中蜀人房審權病易說多岐}\左小列{摘取專明人事者由鄭玄迄王安石凡一百家共}}
\缩进[2]\双列{\右小列{為一百卷名周易義海衡病其蕪雜重複乃刪掇}\左小列{精要以成此書故名曰撮要}}
南軒易說三卷
\缩进[2]\双列{\右小列{宋張栻撰原本十一卷此本出自曹溶家上下經}\左小列{全佚惟存繫辭然繫辭傳托始於天一地二一章}}
\缩进[2]\双列{\右小列{亦非完本蓋元人刊本以程子易傳缺繫辭割栻}\左小列{書補之後又佚其前半也}}
復齋易說六卷
\缩进[2]\双列{\右小列{宋趙彥肅撰其說推尋卦畫即象數以求其理朱}\左小列{子語錄頗病其取義太密然研索於易中完勝支}}
\缩进[2]\双列{\右小列{離於易外也}\左小列{}}
楊氏易傳二十卷
\缩进[2]\双列{\右小列{宋楊簡撰簡為陸九淵之弟子故其說易略象數}\左小列{而談心性多}}
\缩进[2]\双列{\右小列{入於禪錄存其書見以佛理詁易自斯人始著經}\左小列{學別派之由也}}
周易玩詞十六卷
\缩进[2]\双列{\右小列{宋項安世撰前有自述稱其學以伊川易傳為宗}\左小列{然立說頗與伊川異蓋伊川務闡義理安世則兼}}
\缩进[2]\双列{\右小列{言象數以補所遺故與尺寸步趨者殊焉}\左小列{}}
趙氏易說四卷
\缩进[2]\双列{\右小列{宋趙善譽撰舊本二卷久佚今從永樂大典錄出}\左小列{釐為四卷其書推畫卦命名之意以貫通六爻之}}
\缩进[2]\双列{\右小列{旨於諸卦取義相似者參互以盡其變往往具有}\左小列{精理}}
誠齋易傳二十卷
\缩进[2]\双列{\右小列{宋楊萬里撰其書大旨本程氏而參引史傳以證}\左小列{之則與李光之書相同講學家如吳澄陳櫟胡一}}
\缩进[2]\双列{\右小列{桂等皆不滿之蓋門戶之見不足據也}\左小列{}}
大易粹言十卷
\缩进[2]\双列{\右小列{宋方聞一編宋史藝文志作曾穜者誤也是書採}\左小列{二程子張子楊時游酢郭忠孝郭雍七家之說皆}}
\缩进[2]\双列{\右小列{程氏之宗派知其以洛學為主矣}\左小列{}}
易圖說三卷
\缩进[2]\双列{\右小列{宋吳仁傑撰其說以六十四正卦伏羲所作卦外}\左小列{六爻及六十四複卦文王所作又謂序卦為伏羲}}
\缩进[2]\双列{\右小列{作雜卦為文王作今之爻辭當為系辭傳系辭傳}\左小列{當為說卦傳皆故為異說宋人舊帙姑存備一解}}
\缩进[2]\双列{\右小列{云爾}\左小列{}}
古周易一卷
\缩进[2]\双列{\右小列{宋呂祖謙編自王弼以後周易皆以傳附經呂大}\左小列{防以下諸家互有考定而小有異同祖謙乃以上}}
\缩进[2]\双列{\右小列{下經十翼各為一篇復古本之舊朱子本義即用}\左小列{此本也}}
易傳燈四卷
\缩进[2]\双列{\右小列{是書世無傳本諸家書目皆不著錄永樂大典收}\左小列{之題曰宋徐總幹撰亦不著其名惟據原序知為}}
\缩进[2]\双列{\右小列{呂祖謙之門人耳其以釋氏傳燈命名頗為乖剌}\左小列{參以五行家言亦為駁雜然其八卦總論十六篇}}
\缩进[2]\双列{\右小列{參互以求頗能得易之類例}\左小列{}}
易裨傳二卷
\缩进[2]\双列{\右小列{宋林至撰上卷凡三篇一曰法象一曰極數一曰}\左小列{觀變下卷題曰外篇反對相生世應互體納甲變}}
\缩进[2]\双列{\右小列{爻動爻卦氣八事}\左小列{}}
厚齋易學五十二卷
\缩进[2]\双列{\右小列{宋馮椅撰舊本散佚今從永樂大典錄出從其自}\左小列{序釐為輯注四卷輯傳三十卷外傳十八卷輯注}}
\缩进[2]\双列{\右小列{惟解彖象輯傳則以彖象為經而十翼為傳外傳}\左小列{則以十翼為經各附先儒之說而斷以己意}}
童溪易傳三十卷
\缩进[2]\双列{\右小列{宋王宗傳撰其說力排象數而不免涉於虛無大}\左小列{旨與楊簡相類二人同時未知孰倡孰和也}}
周易總義二十卷
\缩进[2]\双列{\右小列{宋易祓撰祓本蘇師旦之黨人不足道然其說易}\左小列{兼該理數折中眾論每卦先括為總論復於六爻}}
\缩进[2]\双列{\右小列{之下詳為詮釋於經義乃頗有發明}\左小列{}}
西溪易說十二卷
\缩进[2]\双列{\右小列{宋李過撰其書首為序說一卷次詮釋經文而不}\左小列{及繫辭以下胡一桂譏其於經文多所竄亂馮椅}}
\缩进[2]\双列{\右小列{則稱其多所發明蓋瑕瑜不掩之書也}\左小列{}}
丙子學易編一卷
\缩进[2]\双列{\右小列{宋李心傳撰書成於嘉定丙子因以為名所取惟}\左小列{王弼張子程子郭雍朱子五家之說而以其父舜}}
\缩进[2]\双列{\右小列{臣之說證之亦間附以己意原本十五卷歲久散}\左小列{佚此本乃宋末俞琰所節抄略存梗概而已}}
易通六卷
\缩进[2]\双列{\右小列{宋趙以夫撰或以為莆田黃績所代作趙汝騰至}\左小列{見彈章莫能詳也大旨以不易變易二義參互以}}
\缩进[2]\双列{\右小列{明人事動靜之準}\左小列{}}
易象意言一卷
\缩进[2]\双列{\右小列{宋蔡淵撰原本久佚今從永樂大典錄出淵元定}\左小列{之子而從學於朱子故此書闡發名理多從師說}}
\缩进[2]\双列{\右小列{兼言數學則本其家傳其兼用互體則取裁古義}\左小列{與講學家持論又殊}}
周易經傳訓解二卷
\缩进[2]\双列{\右小列{宋蔡淵撰原本四卷今佚其二卷惟存上經下經}\左小列{其經文以大象置卦辭下以彖傳置大象後以小}}
\缩进[2]\双列{\右小列{象置爻辭後皆低一字以別卦爻與舊本小異其}\左小列{訓釋則明義理者居多}}
周易要義十卷
\缩进[2]\双列{\右小列{宋魏了翁撰其九經要義之一也即孔穎達周易}\左小列{正義刪繁舉要以便循覽體例頗為簡當}}
東谷易翼傳二卷
\缩进[2]\双列{\右小列{宋鄭汝諧撰所謂翼傳者翼伊川易傳也然於程}\左小列{子之說亦時有異同蓋糾正其失補苴其闕亦所}}
\缩进[2]\双列{\右小列{以羽翼之可謂無朋黨之私矣}\左小列{}}
文公易說二十三卷
\缩进[2]\双列{\右小列{宋朱鑒編鑑為朱子之長孫是書裒輯朱子平日}\左小列{論易之語見於語錄文集者共為一編以發明本}}
\缩进[2]\双列{\右小列{義之旨}\左小列{}}
易學啟蒙小傳一卷
\缩进[2]\双列{\右小列{宋稅與權撰朱子作易學啟蒙多發明邵子先天}\左小列{圖義至於後天之易則以為不得文王所以安排}}
\缩进[2]\双列{\右小列{之意與權研求邵子之說知易有不易之八卦為}\左小列{乾有互易之五十六卦為用反複觀之上下經皆}}
\缩进[2]\双列{\右小列{十八卦羲文之易似異而同因作此書以補朱子}\左小列{之所遺}}
周易輯聞六卷附易雅一卷筮宗一卷
\缩进[2]\双列{\右小列{宋趙汝楳撰周易輯聞但解上下經多所發揮惟}\左小列{竄亂經文是其一失易雅總釋名義凡十八篇如}}
\缩进[2]\双列{\右小列{爾雅之釋詩故名曰雅筮宗凡三篇其中推闡大}\左小列{衍之數頗為明皙}}
周易詳解十六卷
\缩进[2]\双列{\右小列{宋李杞撰以易為有用之學故名用易自序甚明}\左小列{焦竑經籍志作周易詳解者誤也原本二十卷久}}
\缩进[2]\双列{\右小列{已散佚今從永樂大典錄出編為十六卷其書多}\左小列{證以史事與李光楊萬里書同惟頗參以老莊之}}
\缩进[2]\双列{\右小列{說不免駁雜}\左小列{}}
淙山讀周易記二十一卷
\缩进[2]\双列{\右小列{宋方實孫撰其說多主爻象不涉虛無其易卦變}\左小列{合圖補朱子啟蒙所未備}}
周易傳義附錄十四卷
\缩进[2]\双列{\右小列{宋董楷撰以程子之傳朱子之本義合為一書又}\左小列{博採程朱之說附錄其下使互相發明惟割裂本}}
\缩进[2]\双列{\右小列{義以附程傳自楷此書始舊傳始於胡廣等修周}\左小列{易大全者非也}}
易學啟蒙通釋二卷
\缩进[2]\双列{\右小列{宋胡方平撰是書發明朱子易學啟蒙之義所採}\左小列{諸說蔡淵等六家皆朱子之門人蔡模即淵之子}}
\缩进[2]\双列{\右小列{徐幾翁詠又皆淵之門人所謂一家之學也}\左小列{}}
三易備遺十卷
\缩进[2]\双列{\右小列{宋朱元升撰首為河圖洛書一卷祖劉牧之說次}\左小列{連山三卷以卦位配夏時之節氣次歸藏三卷以}}
\缩进[2]\双列{\右小列{干支納音配卦爻次周易三卷皆發反對互體之}\左小列{旨}}
周易集說四十卷
\缩进[2]\双列{\右小列{宋俞琰撰琰初裒諸家易說為大易會要一百三}\左小列{十卷後乃掇其精華以成是書初惟主程朱之說}}
\缩进[2]\双列{\右小列{後乃研索經文浚發新義自為一家之言}\左小列{}}
讀易舉要四卷
\缩进[2]\双列{\右小列{宋俞琰撰琰所著說易之書凡十一種今多散佚}\左小列{此書乃從永樂大典錄出者也琰說易多主朱子}}
\缩进[2]\双列{\右小列{而此書論剛柔往來不主朱子卦變之說其易圖}\左小列{多主邵子而此書論元亨利貞不主起數於四之}}
\缩进[2]\双列{\右小列{說亦可謂不苟異不苟同矣}\左小列{}}
易象義十六卷
\缩进[2]\双列{\右小列{宋丁易東撰是書因象以明義故曰象義其取象}\左小列{之例凡十有二大抵以李鼎祚朱震二家為宗而}}
\缩进[2]\双列{\右小列{卦變則取朱子變卦則取都潔沈該筮占則取朱}\左小列{子蔡淵馮椅亦不偏主於二家}}
易圖通變五卷易筮通變三卷
\缩进[2]\双列{\右小列{宋雷思齊撰其易圖通變以八卦配河圖天一至}\左小列{地八而以五十為虛數與先儒之說頗異其易筮}}
\缩进[2]\双列{\右小列{通變分五篇亦多自出新意蓋奇偶相生變化不}\左小列{窮隨意錯綜無不可以成理也}}
讀易私言一卷
\缩进[2]\双列{\右小列{元許衡撰是書論六爻之德位大旨多發明繫辭}\左小列{傳同功異位柔危剛勝之義其謂各卦畫之居六}}
\缩进[2]\双列{\右小列{位者吉凶悔吝視乎其時而歸於正而得中又彖}\左小列{傳當位不當位得中行中之義也}}
易本義附錄纂注十五卷
\缩进[2]\双列{\右小列{元胡一桂撰是書以朱子本義為宗取朱子文集}\左小列{語錄之說易者附之謂之附錄又纂諸儒之說不}}
\缩进[2]\双列{\右小列{悖於本義者謂之纂注蓋宋末元初朱子之學盛}\左小列{行儒者惟守一先生之言矣}}
易學啟蒙翼傳四卷
\缩进[2]\双列{\右小列{元胡一桂撰一桂之父方平嘗作易學啟蒙通釋}\左小列{一桂更推闡辨別之故曰翼傳凡為內篇者三皆}}
\缩进[2]\双列{\右小列{發朱子占筮圖書之說為外篇者一皆雜論易學}\左小列{之支流}}
易纂言十卷
\缩进[2]\双列{\右小列{元吳澄撰澄於諸經多臆為竄亂惟此經所改大}\左小列{抵依據先儒較為有本其注釋經義亦詞簡而理}}
\缩进[2]\双列{\右小列{明}\左小列{}}
易纂言外翼八卷
\缩进[2]\双列{\右小列{元吳澄撰澄所作易纂言義例散見各卦不相統}\左小列{貫卷首所列卦圖亦粗具梗概乃復作此書暢明}}
\缩进[2]\双列{\右小列{之凡十二篇原本久佚今從永樂大典錄出尚缺}\左小列{其卦變變卦互卦三篇易流易原二篇亦缺其半}}
\缩进[2]\双列{\右小列{然大旨亦可睹矣}\左小列{}}
易源奧義一卷周易原旨六卷
\缩进[2]\双列{\右小列{元寶巴撰案寶巴原本作保八今改正是書原名}\左小列{易體用分為三種今佚其周易尚占三卷僅存其}}
\缩进[2]\双列{\右小列{二大旨皆祖述程朱}\左小列{}}
周易程朱傳義折衷三十三卷
\缩进[2]\双列{\右小列{元趙采撰是書節錄程子易傳朱子本義之文益}\左小列{以語錄諸書而各以己說附於後所註惟上下經}}
\缩进[2]\双列{\右小列{或以程子未注系辭以下故也大旨雖宗宋學而}\左小列{於象數變互尚頗存古義所謂折衷者殆在是歟}}
周易衍義十六卷
\缩进[2]\双列{\右小列{元胡震撰其子廣大四庫總目作光大凡兩見續}\左小列{成之於經文次序臆為顛倒殊嫌乖剌其雜引史}}
\缩进[2]\双列{\右小列{事亦稍傷泛濫然持論尚不失為醇正}\左小列{}}
易學濫觴一卷
\缩进[2]\双列{\右小列{元黃澤撰其說易以明象為本其明象以序卦為}\左小列{本其占法則以左傳為主大旨不取王弼之玄虛}}
\缩进[2]\双列{\右小列{亦不取漢儒之附會故折中以酌其平其歷陳易}\左小列{學不能復古者十三事亦具有根據}}
大易緝說十卷
\缩进[2]\双列{\右小列{元主申子撰前二卷論數學於陳邵諸家之說概}\左小列{斥其有誤其所取者自河圖洛書外惟伏羲文王}}
\缩进[2]\双列{\右小列{周公孔子周子五人未免好為高論然自三卷以}\左小列{下詮釋經文仍以辭變象占乘承比應為說又未}}
\缩进[2]\双列{\右小列{嘗不平正切實}\左小列{}}
周易本義通釋十二卷
\缩进[2]\双列{\右小列{元胡炳文撰大旨以朱子本義為宗而參以眾說}\左小列{原本殘缺惟上下經僅存其十翼乃炳文九世孫}}
\缩进[2]\双列{\右小列{珙玠雜採他書所引炳文之說以補之也}\左小列{}}
周易本義集成十二卷
\缩进[2]\双列{\右小列{元熊良輔撰是書成於延佑復科舉之後元制程}\左小列{試易用程氏朱氏而亦兼用古注疏故是書雖以}}
\缩进[2]\双列{\右小列{羽翼本義為主而亦不盡墨守本義焉}\左小列{}}
大易象數鉤深圖三卷
\缩进[2]\双列{\右小列{元張理撰其書皆即陳摶邵子之說推廣成圖朱}\左小列{子所謂易外別傳者是也}}
學易記九卷
\缩进[2]\双列{\右小列{元李簡撰是書仿李鼎祚集解房審權義海之例}\左小列{採子夏易傳以下六十四家之說亦間附以己意}}
\缩进[2]\双列{\右小列{諸家之書今十不存一其佚文惟賴此書以存}\左小列{}}
周易集傳八卷
\缩进[2]\双列{\右小列{元龍仁夫撰每卦之下各分象變辭占雖大旨根}\左小列{據程朱而於卦象爻象反複推闡頗能自抒心得}}
\缩进[2]\双列{\右小列{故元史稱其發前儒所未發原書十八卷今佚十}\左小列{卷然上下經彖象傳皆尚完具未可以殘缺廢也}}
讀易考原一卷
\缩进[2]\双列{\右小列{元蕭漢中撰是書凡三篇一論分卦一論合卦一}\左小列{論序卦不敢顯攻序卦傳而亦不用序卦之說大}}
\缩进[2]\双列{\右小列{旨雖亦出陳邵而推衍頗有精理尚不失為依經}\左小列{立義}}
易精蘊大義十二卷
\缩进[2]\双列{\右小列{元解蒙撰原本散佚今從永樂大典錄出其例於}\左小列{彖爻之下採輯舊說末乃發明以己意雖為程試}}
\缩进[2]\双列{\右小列{而作然薈萃群言頗有持擇所自注亦皆簡明}\左小列{}}
易學變通六卷
\缩进[2]\双列{\右小列{元曾貫撰原本散佚今從永樂大典錄出其例每}\左小列{篇統論一卦六爻之義又舉他卦辭義之相近者}}
\缩进[2]\双列{\右小列{參互以求異同之故頗為融貫其兼取互體亦能}\左小列{存古義}}
周易會通十四卷
\缩进[2]\双列{\右小列{元董真卿撰真卿受業於胡一桂此書即因一桂}\左小列{纂疏而廣之然一桂堅持門戶真卿則謂諸家之}}
\缩进[2]\双列{\右小列{易途雖殊而歸則同故兼採象數義理兩家以持}\左小列{其平即蘇軾林栗之書朱子所不取者亦不掩其}}
\缩进[2]\双列{\右小列{長則所見視其師為廣矣}\左小列{}}
周易圖說二卷
\缩进[2]\双列{\右小列{元錢義方撰是書凡二十七圖大抵衍陳邵之緒}\左小列{餘然如謂繫辭兼言河圖洛書乃言其理相通非}}
\缩进[2]\双列{\右小列{據洛書以作易又謂陳摶因易而演圖非伏羲據}\左小列{圖以畫卦皆篤論也}}
周易爻變義蘊四卷
\缩进[2]\双列{\右小列{元陳應潤撰大旨謂王弼所注乃老莊虛渺之談}\左小列{陳摶所圖乃參同契爐火之術均非易之本旨又}}
\缩进[2]\双列{\右小列{謂周子太極圖別自一家之說不可以釋易皆能}\左小列{不域於門戶所惟六十四卦其曰爻變即衍左傳}}
\缩进[2]\双列{\右小列{某卦之某卦之古義其謂一卦可變六十四卦亦}\左小列{焦京舊法也}}
周易參義十二卷
\缩进[2]\双列{\右小列{元梁寅撰其說皆即日用常行之事以示進退得}\左小列{失之機頗為平易近人勝於諸家之}}
周易文詮四卷
\缩进[2]\双列{\右小列{元趙汸撰汸於易學不及春秋之深邃然原本宋}\左小列{儒詮釋義理於進退存亡之故吉凶悔吝之理推}}
\缩进[2]\双列{\右小列{闡頗明與梁寅書皆切於人事者也}\左小列{}}
周易傳義大全二十四卷
\缩进[2]\双列{\右小列{明永樂中翰林院學士胡廣等奉敕撰其書魯莽}\左小列{而成僅割裂董楷董真卿胡一桂胡炳文四家之}}
\缩进[2]\双列{\右小列{書餖飣成編以其為一代取士之制故錄之以見}\左小列{經學盛衰之由焉}}
易經蒙引十二卷
\缩进[2]\双列{\右小列{明蔡清撰清篤信朱子之學故是書體例以本義}\左小列{與經文並書但每條之首加一圈以示別然其立}}
\缩进[2]\双列{\右小列{說乃或與本義異同蓋研索者深故一一明其得}\左小列{失猶陸游謂朱子尊程子而說易乃與程子傳異}}
\缩进[2]\双列{\右小列{同也}\左小列{}}
讀易餘言五卷
\缩进[2]\双列{\右小列{明崔銑撰凡上下經卦略二卷大象說繫辭輯說}\左小列{卦訓各一卷大旨以程傳為主而兼採王弼吳澄}}
\缩进[2]\双列{\右小列{之說不甚依附本義論多切實惟點竄說卦而刪}\左小列{除序卦文言未免勇於改經耳}}
易學啟蒙意見五卷
\缩进[2]\双列{\右小列{明韓邦奇撰凡五篇前四篇皆推衍邵子朱子之}\左小列{緒論末一篇曰七占凡六爻不變六爻俱變及一}}
\缩进[2]\双列{\右小列{爻變者皆仍舊法其二爻三爻四爻五爻變者則}\左小列{邦奇所立之新法也}}
易經存疑十二卷
\缩进[2]\双列{\右小列{明林希元撰是書繼蔡清蒙引而作然小有異同}\左小列{大旨為科舉而設故謂漢學不可行於今後來坊}}
\缩进[2]\双列{\右小列{刻講章此其濫觴然明白篤實終非後來講章所}\左小列{及也}}
周易辨錄四卷
\缩进[2]\双列{\右小列{明楊爵撰是書乃嘉靖乙巳爵以建言下詔獄時}\左小列{所作注惟六十四卦經文但載卦辭然注乃併解}}
\缩进[2]\双列{\右小列{六爻彖傳象傳其說多明人事頗為剴切}\左小列{}}
易象鈔四卷
\缩进[2]\双列{\右小列{明胡居仁撰自序稱讀易二十年有所得輒抄積}\左小列{之後二卷則皆與人論易之語及自記所學併為}}
\缩进[2]\双列{\右小列{栝歌詞以舉其要考萬歷乙酉御史李頤請以居}\左小列{仁從祀疏稱所著易傳已散佚此本或後人所裒}}
\缩进[2]\双列{\右小列{輯歟}\左小列{}}
周易象旨決錄七卷
\缩进[2]\双列{\右小列{明熊過撰據其自序蓋因蔡清蒙引陳義而不及}\左小列{象故作此書名決錄者猶言定本也其說遠溯漢}}
\缩进[2]\双列{\右小列{學雖未必遽追梁孟然義必考古終勝明人幻渺}\左小列{之談}}
易象鉤解四卷
\缩进[2]\双列{\右小列{明陳士元撰其說謂易以卜筮為用卜筮以象為}\左小列{宗雖或涉穿鑿然犂然有當者居多惟謂言象為}}
\缩进[2]\双列{\右小列{京房之學則殊舛誤京氏易乃納甲飛伏之學非}\左小列{以象為占也}}
周易集注十六卷
\缩进[2]\双列{\右小列{明來知德撰乃其空山獨處研思二十九年而成}\左小列{專取繫辭錯綜其數之說以錯卦綜卦論易象其}}
\缩进[2]\双列{\右小列{注皆先釋象義字義及錯綜義然後訓本卦本爻}\左小列{正義頗傷繁碎而亦自成一家之學}}
讀易紀聞六卷
\缩进[2]\双列{\右小列{明張獻翼撰是書但隨筆札記不載經文其為人}\左小列{蕩檢踰閑殆有狂疾而說易乃篤實不支多得聖}}
\缩进[2]\双列{\右小列{人示戒之旨蓋其早年力學猶未放誕時作也}\左小列{}}
葉八白易傳十六卷
\缩进[2]\双列{\右小列{明葉山撰八白其字也惟釋六十四卦爻辭大旨}\左小列{以誠齋易傳為宗出入子史佐以博辨蓋借易以}}
\缩进[2]\双列{\右小列{言人事不必盡為經義之所有然所言亦往往可}\左小列{昭法戒}}
洗心齋讀易述十七卷
\缩进[2]\双列{\右小列{明潘士藻撰每條皆先發己意而採掇舊說列於}\左小列{後焦竑序稱所採舊說惟李氏集解房氏義海二}}
\缩进[2]\双列{\右小列{書今觀所引房書較多於李書蓋李書主象漢學}\左小列{之遺房書主理宋學之總士藻所主者宋學也}}
周易像象管見九卷
\缩进[2]\双列{\右小列{明錢一本撰一本所著象抄六卷推衍陳摶之學}\左小列{支離殊無可觀此書作於象抄之前惟即卦爻以}}
\缩进[2]\双列{\右小列{求象即象以明人事雖間有支蔓而篤實近理者}\左小列{多}}
周易札記三卷
\缩进[2]\双列{\右小列{明逯中立撰是書不載經文但以卦名篇名為標}\左小列{識採舊說者十之六出新義者十之四大旨以義}}
\缩进[2]\双列{\右小列{理為主而複姤中孚諸卦亦兼用六日七分之說}\左小列{}}
周易易簡說三卷
\缩进[2]\双列{\右小列{明高攀龍撰其詮解易義每條不過數言故名曰}\左小列{易簡亦頗闡明心學然主於學易以檢心非如楊}}
\缩进[2]\双列{\右小列{簡王宗傳輩引易歸心又引心歸禪也}\左小列{}}
易義古象通八卷
\缩进[2]\双列{\右小列{明魏濬撰前有明象總論八篇大旨謂文周之易}\左小列{即象著理孔子之易以理明象因取漢魏晉唐諸}}
\缩进[2]\双列{\右小列{儒所論象義取其近正者錄之故名曰易義古象}\左小列{通言即象以通義也}}
周易像象述五卷
\缩进[2]\双列{\右小列{明吳桂森撰乃踵其師錢一本像象管見而作故}\左小列{以述為名首列像象金鍼一篇標舉大旨卷中所}}
\缩进[2]\双列{\右小列{注皆一字一句推尋義理頗有新意}\左小列{}}
易用六卷
\缩进[2]\双列{\右小列{明陳祖念撰祖念陳第子也學不逮其父而此書}\左小列{則勝其父伏羲圖贊遠甚書中不載經文但每卦}}
\缩进[2]\双列{\右小列{每章詳論其義務以切於人事為主故名曰用每}\左小列{卦之末總論取象之義多取互體蓋於漢學宋學}}
\缩进[2]\双列{\右小列{無所偏主云}\左小列{}}
易象正十六卷
\缩进[2]\双列{\右小列{明黃道周撰於每卦六爻皆即之卦以觀其變蓋}\左小列{即左傳所載之古法前列日次一卷用漢人分爻}}
\缩进[2]\双列{\右小列{直日之法按文王卦序以推世運後二卷以河圖}\左小列{洛書自相乘除推為三十五圖則均易外之別傳}}
\缩进[2]\双列{\右小列{矣}\左小列{}}
~\newline
\缩进[4]\双列{\右小列{謹按此書及三易洞璣皆皇極經世之支流}\左小列{三易洞璣全推衍於易外故入之術數類此}}
\缩进[4]\双列{\右小列{及倪元璐兒易有於易外者猶有據經立義}\左小列{發揮於易中者且皆忠節之士宜因人以重}}
\缩进[4]\双列{\右小列{其書故此二編仍著錄於經部非通例也}\左小列{}}
兒易內儀以六卷兒易外儀十五卷
\缩进[2]\双列{\右小列{明倪元璐撰名兒易者據元璐自序蓋取孩始之}\左小列{義其內儀以專以大象釋經以六十四卦大象皆}}
\缩进[2]\双列{\右小列{有以字故以為名也外儀分六目六目又各分子}\左小列{目皆以繫辭中字義名篇篇各有圖大抵憂時傷}}
\缩进[2]\双列{\右小列{亂借易以抒其意不必盡為經義之所有}\左小列{}}
卦變考略一卷
\缩进[2]\双列{\右小列{明董守諭撰以朱子卦變圖與本義自相矛盾因}\左小列{考郎京房蜀才虞翻諸家之說推衍成圖以存古}}
\缩进[2]\双列{\右小列{義}\左小列{}}
古周易訂詁十六卷
\缩进[2]\双列{\右小列{明何楷撰前六卷以傳附經用王弼本七卷以下}\左小列{則仍以十翼原文存田何之舊其學雖博而不精}}
\缩进[2]\双列{\右小列{然取材宏富詞必有據漢晉以來之古義頗藉以}\左小列{見梗概}}
周易玩詞困學記十五卷
\缩进[2]\双列{\右小列{明張次仲撰自序謂不敢侈談象數又雅不信讖}\左小列{緯之說惟於語言文字間求其有益於身心者持}}
\缩进[2]\双列{\右小列{論頗為篤實其鏟除諸圖亦具有廓清之力}\左小列{}}
易經通註九卷
\缩进[2]\双列{\右小列{國朝大學士傅以漸等奉}\左小列[indent=-1]{敕撰順治十三年十二月}}
\缩进[-1]\双列{\右小列{世祖章皇帝以永樂易經大全繁而可刪華而寡要因}\左小列{命以漸等刊其舛訛補其缺漏勒為是書以順治十五年}}
\缩进[-1]\双列{\右小列{十月告成}\左小列{}}
日講易經解義十八卷
\缩进[2]\双列{\右小列{康熙二十二年大學士牛鈕等奉}\左小列[indent=-1]{敕編用宋代經筵講義之體發揮要旨疏通證明不取莊}}
\缩进[-1]\双列{\右小列{老之虛無亦不取焦京之術數惟即辭占象變敷陳人事}\左小列{以明法天建極之實功故}}
\缩进[-1]\双列{\右小列{御製序文特揭以經學為治法之義焉}\左小列{}}
\单抬 御纂周易折中二十二卷
\缩进[2]\双列{\右小列{康熙五十四年大學士李光地等奉}\左小列[indent=-1]{敕撰自董楷析朱子周易本義附於程傳十二篇舊第復}}
\缩进[-1]\双列{\右小列{淆是編恭稟}\左小列{聖裁改從古本足正千古之訛大旨雖根據程朱而參考}}
\缩进[-1]\双列{\右小列{群言務求至當實不偏主一家允為說易之准繩}\左小列{}}
御纂周易述義十卷
\缩进[2]\双列{\右小列{乾隆二十年大學士傅恒等奉}\左小列[indent=0]{敕撰以本}}
\缩进[-1]\双列{\右小列{御纂周易折中而推闡之故名述義大旨謂易因人事以}\左小列{立象故不涉虛渺之說與術數之學其觀象多取於互體}}
\缩进[-1]\双列{\右小列{尤能發明古義漢易宋易至是而集其成矣}\左小列{}}
讀易大旨五卷
\缩进[2]\双列{\右小列{國朝孫奇逢撰皆其讀易有得錄示門人之語其}\左小列{說不顯攻圖書亦無一字及圖書惟以象傳通一}}
\缩进[2]\双列{\右小列{卦之旨以一卦通六十四卦之義皆切近人事發}\左小列{明義理末附兼山堂問答及與李崶論易之語別}}
\缩进[2]\双列{\右小列{為一卷崶即奇逢所從受易者也}\左小列{}}
周易稗疏四卷附考異一卷
\缩进[2]\双列{\右小列{國朝王夫之撰皆隨筆札記以剖析疑義大旨不}\左小列{信焦京亦不信陳邵亦不取王弼之清言惟引據}}
\缩进[2]\双列{\右小列{訓詁考求古義所謂徵實之學也}\左小列{}}
易酌十四卷
\缩进[2]\双列{\右小列{國朝刁包撰大旨以程子傳朱子本義為主雖亦}\左小列{兼言象數然皆陳摶李之才之學非漢以來之舊}}
\缩进[2]\双列{\右小列{學也取其持論篤實而已}\左小列{}}
田間易學十二卷
\缩进[2]\双列{\右小列{國朝錢澄之撰澄之初問易於黃道周故頗詳於}\左小列{數學後乃兼求義理參取於王弼孔穎達程子朱}}
\缩进[2]\双列{\右小列{子之間其謂先天河洛皆因易而作圖用錢義方}\左小列{之說謂圖中奇偶乃揲蓍之法非畫卦之本用陳}}
\缩进[2]\双列{\右小列{應潤之說也}\左小列{}}
易學象數論六卷
\缩进[2]\双列{\右小列{國朝黃宗羲撰宗羲以易至焦京而流為方術至}\左小列{陳摶而歧入道家九流百氏罔弗依託因作此以}}
\缩进[2]\双列{\右小列{糾其失前三卷論河圖洛書先天方位納甲納音}\左小列{月建卦氣卦變互卦筮法占法附以所作原象為}}
\缩进[2]\双列{\右小列{內篇後三卷論太玄乾鑿度元包潛虛洞極洪範}\左小列{數皇極數以及六壬太乙遁甲為外篇}}
周易象辭二十一卷附尋門餘論二卷圖書辨惑一卷
\缩进[2]\双列{\右小列{國朝黃宗炎撰宗炎力闢陳摶之學故所解惟主}\左小列{義理然根據經典不涉空談尋門餘論兼排釋氏}}
\缩进[2]\双列{\右小列{未免蔓衍於易外而其他持論多醇正圖書辨惑}\左小列{論先天圖與陳應潤所言合論太極圖與朱彞尊}}
\缩进[2]\双列{\右小列{毛奇齡所考合亦皆明確也}\左小列{}}
周易筮述八卷
\缩进[2]\双列{\右小列{國朝王宏撰撰以朱子謂易本卜筮之書因作此}\左小列{編以明其義凡十五篇雖端為揲蓍而作然闢焦}}
\缩进[2]\双列{\右小列{京之小術闡羲文周孔之宏旨立論悉本經義與}\左小列{方技家所說迥殊故進之列於易類不以術數論}}
\缩进[2]\双列{\右小列{焉}\左小列{}}
仲氏易三十卷
\缩进[2]\双列{\右小列{國朝毛奇齡撰是書述其兄錫齡之遺說故以仲}\左小列{氏為名大旨謂易兼五義一曰變易一曰交易為}}
\缩进[2]\双列{\右小列{先儒之所知一曰反易一曰對易一曰移易皆先}\左小列{儒之所未知其言甚辨然大致有所根據非純構}}
\缩进[2]\双列{\右小列{虛詞}\左小列{}}
推易始末四卷
\缩进[2]\双列{\右小列{國朝毛奇齡撰奇齡既作仲氏易因採漢以來諸}\左小列{儒之言卦變者別加綜核以成是書其名推易蓋}}
\缩进[2]\双列{\右小列{本繫辭剛柔相推之文即仲氏易所謂移易也}\左小列{}}
春秋占筮書三卷
\缩进[2]\双列{\右小列{國朝毛奇齡撰摭春秋傳所載占筮以明古人之}\左小列{易學實為易作非為春秋作也}}
易小帖五卷
\缩进[2]\双列{\右小列{國朝毛奇齡說易之語其門人記錄成書者也凡}\左小列{一百四十三條與仲氏易互相發明大抵徵引古}}
\缩进[2]\双列{\右小列{義以糾近代說易之失於王弼陳摶二派掊擊尤}\左小列{力}}
喬氏易俟十八卷
\缩进[2]\双列{\右小列{國朝喬萊撰前列諸圖不取陳摶之說於卦變亦}\左小列{不取虞翻諸家之說而取來知德之反對其解經}}
\缩进[2]\双列{\右小列{多推求人事證以史文蓋李光楊萬里之支流也}\左小列{}}
讀易日抄六卷
\缩进[2]\双列{\右小列{國朝張烈撰一以朱子本義為宗因象設事就事}\左小列{陳理猶近時易說之不枝蔓者}}
周易通論四卷
\缩进[2]\双列{\右小列{國朝李光地撰一卷二卷發明上下經大旨三卷}\左小列{四卷發明繫辭說卦序卦雜卦之義冠以易本易}}
\缩进[2]\双列{\右小列{教二篇次論卦爻象彖時位德應河圖洛書以及}\左小列{占筮掛扐正變環互皆一一詳悉其於宋易可謂}}
\缩进[2]\双列{\右小列{融會貫通矣}\左小列{}}
周易觀彖十二卷
\缩进[2]\双列{\右小列{國朝李光地撰是書取繫辭觀其彖辭則思過半}\左小列{之義實注全經非止解彖辭其語錄文集頗申明}}
\缩进[2]\双列{\右小列{先天諸圖此書則惟說卦傳天地定位一章略及}\左小列{斯義餘無一字及之則亦知非畫卦之本矣經中}}
\缩进[2]\双列{\右小列{脫文誤字惟繫辭侯之二字作衍文餘皆不從程}\左小列{傳本義其說皆自抒心得亦不甚附合程朱也}}
周易淺述八卷
\缩进[2]\双列{\右小列{國朝陳夢雷撰乃康熙甲戌夢雷戍尚陽堡時所}\左小列{作大旨主本義而參以王弼孔穎達蘇軾來知德}}
\缩进[2]\双列{\右小列{及永樂大全蓋行篋乏書故所據止此其說多即}\左小列{象以明人事末附三十圖則其友楊道聲作也}}
易原就正十二卷
\缩进[2]\双列{\右小列{國朝包儀撰其學從先天圖入故自序謂皇極經}\左小列{世為易之本旨然每爻注變卦猶用古法詮釋簡}}
\缩进[2]\双列{\右小列{明亦不繳繞奇偶排比黑白與自序實不相應也}\左小列{}}
大易通解十五卷
\缩进[2]\双列{\右小列{國朝魏荔彤撰其論畫卦謂河圖洛書只可云其}\左小列{理相通不必穿鑿附會謂先天圖非生卦之次序}}
\缩进[2]\双列{\右小列{論爻謂當兼變爻謂泰否損益四卦為上下經之}\左小列{樞紐皆具有理解惟不取扶陽抑陰之說則未審}}
\缩进[2]\双列{\右小列{姤復之初爻矣}\左小列{}}
易經衷論二卷
\缩进[2]\双列{\右小列{國朝張英撰所釋惟六十四卦每卦為論一篇其}\左小列{立說主於顯易不務艱深頗能掃眾說之糾結}}
易圖明辨十卷
\缩进[2]\双列{\右小列{國朝胡渭撰其一卷辨河圖洛書二卷辨五行九}\左小列{宮三卷辨參同契先天圖太極圖四卷辨龍圖易}}
\缩进[2]\双列{\右小列{數鉤隱圖五卷辨啟蒙圖書六卷七卷辨先天古}\左小列{易八卷辨後天之學九卷辨卦變十卷辨象數流}}
\缩进[2]\双列{\右小列{弊並引據經典元元本本於易學深為有功}\左小列{}}
合訂刪補大易集義粹言八十卷
\缩进[2]\双列{\右小列{國朝納喇性德撰是書取宋陳友文大易集義方}\左小列{聞一大易粹貫案原本訛方聞一為曾穜今考正}}
\缩进[2]\双列{\右小列{刪除重複刊削繁蕪合為一編宋儒易說略具於}\左小列{斯}}
周易傳註七卷附周易筮考一卷
\缩进[2]\双列{\右小列{國朝李塨撰其說以易卦本以人事立言陳摶劉}\左小列{牧諸圖皆使易道入於無用參同契三易洞璣之}}
\缩进[2]\双列{\右小列{類皆以異端方技闌入經學即漢儒卦氣直日之}\左小列{類亦經外別生枝節故惟以觀象為主第不廢互}}
\缩进[2]\双列{\右小列{體耳}\左小列{}}
周易札記二卷
\缩进[2]\双列{\右小列{國朝楊名時撰名時易學多得之其師李光地是}\左小列{書惟說卦傳及附論啟蒙之類頗推衍先天諸圖}}
\缩进[2]\双列{\右小列{餘皆發揮易理者也}\左小列{}}
周易傳義合訂十二卷
\缩进[2]\双列{\右小列{國朝朱軾撰凡程傳本義互有異同者務折中以}\左小列{歸一使不涉兩岐惟兩義並行不悖者乃兼存其}}
\缩进[2]\双列{\右小列{說附以諸儒之論或有實勝程朱者亦舍程朱以}\左小列{從之蓋不株守門戶之見也}}
周易玩詞集解十卷
\缩进[2]\双列{\右小列{國朝查慎行撰慎行受業於黃宗羲故於易家一}\左小列{切雜學灼然不惑其河圖說卦變說天根月窟考}}
\缩进[2]\双列{\右小列{八卦相錯說闢卦說中爻互體說廣八卦說辨証}\左小列{具有根據詮釋經文亦明切不支}}
惠氏易說六卷
\缩进[2]\双列{\右小列{國朝惠士奇撰其書雜釋卦爻專明漢學大抵以}\左小列{象為主而訓詁尤所加意惟欲矯王弼等空言之}}
\缩进[2]\双列{\右小列{弊採掇未免駁雜然其精核者終不可廢也}\左小列{}}
周易函書約存十八卷約註十八卷別集十六卷
\缩进[2]\双列{\右小列{國朝胡煦撰原本一百十八卷稿本浩繁漸有散}\左小列{佚其已刻者亦編次無緒此本乃其子季堂以其}}
\缩进[2]\双列{\右小列{論易之語分為原圖原卦原爻原占者編為約存}\左小列{以其依經釋義者編為約注而以篝燈約旨易解}}
\缩进[2]\双列{\右小列{辨異易學須知編為別集其持論酌於漢學宋學}\左小列{之間與朱子頗有異同}}
易箋八卷
\缩进[2]\双列{\右小列{國朝陳法撰其書以易為專明人事其駁來知德}\左小列{錯綜之說最為明皙其論筮法亦具有理解}}
楚蒙山房易經解十六卷
\缩进[2]\双列{\右小列{國朝晏斯盛撰凡易學初津二卷不取圖書之說}\左小列{乃併卦變互體而廢之未免主持稍過易翼宗六}}
\缩进[2]\双列{\右小列{卷詮釋經文附以十翼易翼說八卷詮釋十翼又}\左小列{各自為篇與何楷古周易訂詁例同亦嫌繁複然}}
\缩进[2]\双列{\右小列{所解斟酌於言理言數之間則頗能持其平}\左小列{}}
周易孔義集說二十卷
\缩进[2]\双列{\右小列{國朝沈起元撰以孔子十翼為主定眾說之是非}\左小列{前列三圖一曰八卦方位一曰乾坤生六子一曰}}
\缩进[2]\双列{\右小列{因重皆據繫辭說卦其先天諸圖則以為陳邵之}\左小列{易非孔子之易概從芟剃持論特確所解亦多能}}
\缩进[2]\双列{\右小列{推驗舊詁引申新義惟既用王弼散附之本而又}\左小列{以大象文言析出自為一傳則自我作古耳}}
易翼述信十二卷
\缩进[2]\双列{\右小列{國朝王又樸撰其說亦以十翼為主深以朱子所}\左小列{云不可以孔子之易為文王之易者為非其所徵}}
\缩进[2]\双列{\右小列{引惟李光地之說為多亦不甚墨守本義也}\左小列{}}
周易淺釋四卷
\缩进[2]\双列{\右小列{國朝潘思矩撰大旨即象明理而即互體卦變以}\左小列{求象每卦皆註自某卦來謂之時來是亦易中之}}
\缩进[2]\双列{\右小列{一義不足盡易而不可謂之非易固可存備一家}\左小列{也}}
周易洗心九卷
\缩进[2]\双列{\右小列{國朝任啟運撰其說多發明圖學謂論語之五十}\左小列{學易即指河圖之五十立論殊為新異其詮釋經}}
\缩进[2]\双列{\右小列{文則觀象玩詞時標精理其考定文句亦根據先}\左小列{儒然則啟運之講圖學特好語精微耳非如張行}}
\缩进[2]\双列{\右小列{成等竟舍經而言數也}\左小列{}}
豐川易說十卷
\缩进[2]\双列{\右小列{國朝王心敬撰心敬所註諸經皆好為異論是書}\左小列{闡發易理乃取諸人事謂陰陽消長不過借作影}}
\缩进[2]\双列{\右小列{子特為切實惟排斥雜學併左傳占法而詆之為}\左小列{主持太過耳}}
周易述二十三卷
\缩进[2]\双列{\右小列{國朝惠棟撰其書主發揮漢儒之學以荀爽虞翻}\左小列{為主而參以鄭玄宋咸干寶諸家之說自為註而}}
\缩进[2]\双列{\右小列{自疏之凡二十一卷中闕下經一卷又闕序卦雜}\左小列{卦傳蓋未完之本末二卷為易微言雜抄經典論}}
\缩进[2]\双列{\右小列{易之語叢冗無緒亦未及排纂之稿本也}\左小列{}}
易漢學八卷
\缩进[2]\双列{\右小列{國朝惠棟撰考漢易宗派源流掇拾緒論以見大}\左小列{凡凡孟長卿易二卷虞仲翔易一卷京君明易二}}
\缩进[2]\双列{\右小列{卷干寶附焉鄭康成易一卷荀慈明易一卷其末}\左小列{一卷則棟發明漢易之理以辨正河圖洛書先天}}
\缩进[2]\双列{\右小列{太極之學}\左小列{}}
易例二卷
\缩进[2]\双列{\右小列{國朝惠棟撰皆考究漢儒之傳以發明易之本例}\左小列{凡九十類其中有錄無書者十三類所分門目頗}}
\缩进[2]\双列{\右小列{多牽混蓋亦未成之稿然棟於諸經精研古義其}\左小列{所採摭多專門授受之學儻因而排纂猶可見作}}
\缩进[2]\双列{\右小列{易之大綱未可以冗雜棄也}\左小列{}}
易象大意存解一卷
\缩进[2]\双列{\右小列{國朝任陳晉撰多申尚象之旨不載經文惟折中}\左小列{諸家之說明其大意首論太極五行先天河洛皆}}
\缩进[2]\双列{\右小列{鏟除次論彖論爻論象次論六十四卦多指陳法}\左小列{戒終以系辭說卦序卦雜卦其文頗略以所重在}}
\缩进[2]\双列{\右小列{六十四卦也}\左小列{}}
大易擇言三十六卷
\缩进[2]\双列{\右小列{國朝程廷祚撰因桐城方苞緒論以六例編纂諸}\左小列{家之說一曰正義二曰辨正三曰通論四曰餘論}}
\缩进[2]\双列{\右小列{五曰存疑六曰存異大抵力排象數惟以義理為}\左小列{宗}}
周易辨畫四十卷
\缩进[2]\双列{\右小列{國朝連斗山撰大旨謂一卦之義在於爻爻畫有}\左小列{剛有柔因剛柔之畫而立之象即因剛柔之畫而}}
\缩进[2]\双列{\右小列{繫以辭其道先在於辨畫故以為名雖不免或涉}\左小列{穿鑿然逐卦剖析互體亦時有精理}}
周易圖書質疑二十四卷
\缩进[2]\双列{\右小列{國朝趙繼序撰其書以象數言易而不主先天河}\左小列{洛之說首為古經十二篇次逐節詮釋經義而不}}
\缩进[2]\双列{\右小列{載經文蓋用經傳別行之古例次為圖三十有二}\左小列{為說五其詁經多從卦變起象而兼取漢宋之說}}
\缩进[2]\双列{\右小列{無所偏主}\左小列{}}
周易章句證異十一卷
\缩进[2]\双列{\右小列{國朝翟均廉撰皆考究諸本辨周易篇章字句之}\左小列{同異校勘頗為精密附錄}}
乾坤鑿度二卷
\缩进[2]\双列{\右小列{是書為永樂大典所載易緯八種之一分上下二}\左小列{篇上篇論四門四正取象取物以至卦爻蓍策之}}
\缩进[2]\双列{\右小列{數下篇論坤有十性而推及於蕩配凌配又雜引}\左小列{諸緯書之詞佶屈聱牙頗不易曉}}
周易乾鑿度二卷
\缩进[2]\双列{\右小列{是書為易緯八種之二舊本標鄭康成注唐以前}\左小列{說經之家恒相引用其太乙行九宮法即後世洛}}
\缩进[2]\双列{\右小列{書所從出在緯書之中特為醇正}\左小列{}}
易緯稽覽圖二卷
\缩进[2]\双列{\右小列{是書為易緯八種之三首言卦氣起中孚而以坎}\左小列{離震兌為四正卦六十卦卦主六日七分又以自}}
\缩进[2]\双列{\右小列{坤至複十二卦為消息餘雜卦主公卿侯大夫候}\左小列{風雨寒溫以為征應即孟喜京房之學至所稱軌}}
\缩进[2]\双列{\右小列{之數以及世應游歸乃兼通日家推步之法唐一}\左小列{行大衍歷議即演其術惟所紀年號至唐元和疑}}
\缩进[2]\双列{\右小列{術家所附益也}\左小列{}}
易緯辨終備一卷
\缩进[2]\双列{\右小列{是書為易緯八種之四一作辨中備傳寫異文也}\左小列{今永樂大典所載僅數十言似非完本以古來著}}
\缩进[2]\双列{\右小列{錄姑存以備考核}\左小列{}}
易緯通卦驗二卷
\缩进[2]\双列{\右小列{是書為易緯八種之五宋史藝文志作二卷永樂}\左小列{大典所載合為一篇今核其文義定人主動而得}}
\缩进[2]\双列{\右小列{天地之道則萬物之精盡矣以上為上卷曰凡易}\左小列{八卦之氣驗應各如其法度以下為下卷上言稽}}
\缩进[2]\双列{\右小列{應之理下言卦氣之徵驗也}\左小列{}}
易緯乾元序制記一卷
\缩进[2]\双列{\右小列{是書為易緯八種之六唐以前史不著錄陳振孫}\左小列{書錄解題始載之然其文乃與諸書所引是類謀}}
\缩进[2]\双列{\右小列{坤靈圖稽覽圖之文相同疑後人割裂緯書偽題}\左小列{此名也}}
易緯是類謀一卷
\缩进[2]\双列{\右小列{是書為易緯八種之七通體以韻語成文多言禨}\左小列{祥推驗並及於姓輔名號與乾鑿度所引易歷義}}
\缩进[2]\双列{\右小列{相發明}\左小列{}}
易緯坤靈圖一卷
\缩进[2]\双列{\右小列{是書為易緯八種之八殘缺不完僅存論乾大蓄}\左小列{無妄卦辭及史注所引日月合璧數語而已}}
\缩进[1] 右易類共一百五十八部一千七百五十七卷附錄
\缩进[1] 八部十二卷

\end{数字化内容}
\end{document}
