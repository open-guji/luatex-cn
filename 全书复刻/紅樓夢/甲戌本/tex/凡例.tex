\documentclass[红楼梦甲戌本]{ltc-guji}
% \cnvdebugon
% \禁用分页裁剪

\usepackage{fontspec}
\setmainfont{TW-Kai}[CharacterWidth=Full]
% 如果不设置将使用系统默认字体

\title{石頭記}

\chapter{卷一}

\SetPageNumber{4}

\begin{document}
\begin{正文}
脂硯齋重評石頭記

\begin{段落}[indent=1]
凡例
\end{段落}

\begin{段落}[indent=2]
紅樓夢㫖義\空格 是書題名極多\空格[2] 紅樓夢是縂其全部之名也又曰風月寳鑑是戒妄動風月之情又曰石頭記是自譬石頭所記之事也此三名皆書中曾已點睛矣如寳玉做夢夢中有曲名曰紅樓夢十二支此則紅樓夢之點睛又如賈瑞病跛道人持一鏡來上面即鏨風月寳鑑四字此則風月寳鑑之點睛又如道人親眼見石上大書一篇故事則係石頭所記之徃來此則石頭記之點睛處然此書又名曰金陵十二釵審其名則必係金陵十二女子也然通部細搜檢去上中下女子豈止十二人哉若云其中自有十二個則又未嘗指明白係某某極至紅樓夢一回中亦曾翻出金陵十二釵之簿籍又有十二支曲可考

書中凡冩長安在文人筆墨之間則從古之稱凡愚夫婦兒女子家常口角則曰中京是不欲着跡于方向也蓋天子之邦亦當以中為尊特避其東南西北四字様也此書只是着意于閨中故叙閨中之事切略涉於外事者則簡不得謂其不均也

此書不敢干涉朝廷凡有不得不用朝政者只畧用一筆帶出盖實不敢以冩兒女之筆墨唐突朝廷之上也又不得謂其不備

此書開卷第一回也作者自云因曾歴過一番夢幻之後故將真事隠去而撰此石頭記一書也故曰甄士隠夢幻識通靈但書中所記何事又因何而撰是書哉自云今風塵碌碌一事無成忽念及當日所有之女子一一細推了去覺其行止見識皆出于我之上何堂堂之鬚眉誠不若彼一干裙釵實愧則有餘悔則無益之大無可奈何之日也當此時則自欲將已徃所頼上頼天恩下承祖德錦衣紈絝之時飫甘饜羙之日背父母教育之恩負師兄規訓之德已致今日一事無成半生潦倒之罪編述一記以告普天下人雖我之罪固不能免然閨閣中本自歴歴有人萬不可因我不肖則一併使其泯滅也雖今日之茅椽蓬牖瓦竈繩床其風晨月夕堦柳庭花亦未有傷于我之襟懷筆墨者何為不用假語村言敷演出一叚故事來以悅人之耳目哉故曰風塵懷閨秀乃是第一回題綱正義也開卷即云風塵懷閨秀則知作者本意原爲記述當日閨友閨情並非怨世罵時之書矣雖一時有涉于世態然亦不得不叙者但非其本㫖耳閱者切記之
\end{段落}

\begin{段落}[indent=3]
詩
\end{段落}
\begin{段落}[indent=3]
曰
\end{段落}

\begin{段落}[indent=2]
浮生着甚苦奔忙\空格 盛席華筵終散場
悲喜千般同幻渺\空格 古今一夢盡荒唐
謾言紅袖啼痕重\空格 更有情痴抱恨長
字字看來皆是血\空格 十年辛苦不尋常
\end{段落}

\end{正文}
\end{document}
