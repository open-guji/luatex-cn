\documentclass[红楼梦甲戌本]{ltc-guji}
% \禁用分页裁剪

\usepackage{fontspec}
\setmainfont{TW-Kai}
% 如果不设置,将使用系统默认字体

\title{石頭記}

\chapter{卷一}

\SetPageNumber{4}

\begin{document}
\begin{正文}
第一回

\begin{段落}[indent=2]
甄士隱夢幻識\夹注{通靈} 賈雨村風塵懷\夹注{閨秀}
\end{段落}

\批注[x=397pt, y=100pt, height=7]{
    妙自謂落堕情\\根故無補天之用
}
列位看官你道此書從何而來說起根由雖\侧批[yoffset=-10pt]{自占地步}近荒唐細\侧批{自首荒唐妙}諳則深有趣味待在下將此來歷註明方使閱者了然不惑原來女媧氏煉\侧批{補天濟世勿認真用常言}石補天之時于大\侧批[yoffset=7pt]{荒唐也}荒山無稽\侧批[yoffset=-10pt]{無稽也}崖煉成高經\侧批{縂應十二釵}十二丈方經二\侧批{照應副十二釵}十四丈頑石三萬六千五百零一塊媧皇氏只用了三萬六千\侧批{合週天之数}五百塊只\侧批{剩了這一塊便生出這許多故事使當日雖不以此補天就該去補地之坑陷使地平坦而不得有此一部鬼話}单单的剩了一塊未用便棄在此山青埂峰下誰知此石自經\侧批{煆煉後性方通甚哉人生不能學也}煆煉之後靈性已通因見衆石俱得補天獨自己無材不堪入選遂自怨自嘆日夜悲號慚愧一日正當嗟悼之際俄見一僧一道遠遠而來生得

骨格不凡丰神迥別說說笑笑來至峰下坐于石邊高談快論先是說些雲山霧海神僲玄幻之事後便說到紅塵中榮華富貴此石聽了不覺打動凡心也想要到人間去享一享這榮華富貴但自恨粗蠢不得\侧批{竟有人問口生於何處其無心肝可咲可恨之極}已便口吐人言向那僧道說道大師弟子\侧批[yoffset=-10pt]{豈敢豈敢}蠢物不能見禮了適聞二位談那人世間榮耀繁華心切慕之弟子質雖粗蠢性\侧批{豈敢豈敢}却稍通况見二師仙形道體定非凡品必有補天濟世之材利物濟人之德如蒙發一㸃慈心攜帶弟子得入紅塵在那富貴場中溫柔鄉裏受享幾年自當永佩洪恩萬劫不忘也二仙師聽畢齊憨笑道善哉善哉那紅塵中有却

有些樂事但不能永遠依恃况又有羙中不足好事多魔八箇字緊相連屬瞬息間則又樂極悲生人非物換究竟是到頭一夢\侧批{四句乃一部之縂綱}萬境歸空到不如不去的好這石凡心已熾那里聽得進這話去乃復苦求再四二仙知不可強制乃嘆道此亦靜極思動無中生有之數也既如此我們便携你去受享受享只是到不得意時切莫後悔石道自然自然那僧又道若說你性靈却又如此質蠢並更無竒貴之處如此\侧批{煆煉過尚與人踮腳不學者又當如何}也只好踮脚而已也罷我如今大施\侧批{妙佛法亦湏償還况世人之償乎近之頼債者來看此句所謂逰戲茟墨也}佛法助你助待劫終之日復還本質以了此案你道好否石頭聽了感謝不盡那僧便念咒書符大\侧批{明㸃幻字好}展幻術將一塊大

\批注[x=397pt, y=100pt, height=7]{昔子房後謁黃石公惟見一石子房當時恨不能隨此石去余亦恨不能隨此石而去也聊供閱者一笑}
石登時變成一塊鮮明瑩潔的羙玉且又縮成\侧批{奇詭險怪之文有如髯蘇石鐘赤璧用幻處}扇墜大小的可佩可拿那僧托于掌上笑道形體到也是個\侧批{自愧之語}寳物了\侧批{妙極今之金玉其外敗絮其中者見此大不歡喜}還只沒有實在的好處湏得在\侧批{世上原宜假不宜真也諺云一日賣了三千假三日賣不出一個真信哉}鐫上數字使人一見便知是竒物方妙然後好携你到那昌明隆盛之邦\侧批{伏長安大都}詩禮\侧批{伏榮國府}簮□之族花\侧批{伏大觀園}桞繁華地溫柔富貴鄉\侧批{伏紫芸軒}去安身樂業\侧批{何不再添一句云擇個絶世情痴作主人}石頭聽了喜不能禁乃問不知賜了弟子那幾件奇處\侧批{可知若果有奇貴之處自己亦不知者若自以竒貴而居究竟是無真竒貴之人}又不知携了弟子到何地方望乞明示使弟子不惑那僧笑道你且莫問日後自然明白的說着便袖了這石同那道人飄然而去竟不知投奔何方何舍後來不知又過了幾世幾劫因有個空空道人訪道求仙忽從這大荒山無稽崖

青埂峰下經過忽見一大石上字跡分明編述歷歷空空道人乃從頭一看原來就是無材補天幻形入世\侧批{八字便是作者一生慚恨}蒙茫茫大士渺渺真人携入紅塵歷盡離合悲歡炎凉世態的一叚故事後面又有一首偈云

\begin{段落}[indent=2]
無材可去補\侧批{書之本旨}蒼天\空格 枉\侧批{慚愧之言嗚咽如聞}入紅塵若許年
此係身前身後事\空格 倩誰記去作奇傳
\end{段落}

詩後便是此石堕落之鄉投胎之處親自經歷的一叚陳跡故事其中家庭閨閣瑣事以及閑情詩詞倒還全偹或\侧批{或字謙得好}可適趣觧悶然朝代年紀地輿邦國却反失落無考\侧批{若用此套者胸中必無好文字手中断無新茟墨據余說卻大有考證}空空道人遂向石頭說道石兄你這一叚故事據你自己說有些趣

味故編冩在此意欲問世傳奇據我看來第一件無朝代年紀可考\侧批{先駁得妙}第二件並無大賢大忠理朝廷治風俗的善政\侧批{将世人欲駁之腐言預先代人駁盡妙}其中只不過幾箇異樣的女子或情或痴或小才㣲善亦無班姑蔡女之德能我縂抄去恐世人不愛看呢石頭笑荅道我師何太痴耶若云無朝代可考今我師竟假借漢唐等年紀添綴又有何難\侧批{所以答的好}但我想歷來野史皆蹈一轍莫如我這不借此套者反到新奇別致不過只取其事體情理罷了又何必拘拘扵朝代年紀哉再者世井俗人喜看理治之書者甚少愛看適趣閒文者特多歷代野史或訕謗君相或貶人妻女\侧批{先批其大端}姦滛凶惡不可勝数更有


一種風月筆墨其滛穢污臭塗毒筆墨壞人子弟又不可勝数至若佳人才子等書則又千部共出一套且其中終不能不涉于滛濫以致滿紙潘安子建西子文君不過作者要冩出自己的那兩首情詩艶賦來故假擬出男女二人名姓又必傍出一小人其間撥亂亦如劇中之小丑然且嬛婢開口即者也之乎非文即理故逐一看去悉皆自相矛盾大不近情理之話竟不如我半世親睹親聞的這幾個女子雖不敢說強似前代書中所有之人但事跡原委亦可以消愁破悶也有幾首歪詩熟話可以噴飯供酒至若離合悲歡興衰際遇則又追踪攝跡不敢
\批注[x=450pt, y=100pt, height=7]{事則實事然亦叙得有間架有曲折有順逆有映帶有隱有見有正有閏以至草蛇灰線空谷傳聲一擊兩鳴明修棧道暗度陳倉雲龍霧雨兩山對峙烘雲托月背面傳粉千皴萬染諸竒書中之秘法亦不復少余亦于逐回中搜剔刳剖明白注釋以待高明再批示誤謬}


稍加穿鑿徒為供人之目而反失其真傳者今之人貧者日為衣食所累富者又懷不足之心縂一時稍閒又有貪滛戀色好貸尋愁之事那里去有工夫看那理治之書所以我這一叚事也不愿世人稱竒道妙也不定要世人喜悅檢讀\侧批{轉得更好}只愿他們當那醉餘飽卧之時或避世去愁之際把此一玩豈不省了此壽命筋力就比那謀虛逐妄去也省了口舌是非之害腿脚奔忙之苦再者亦令世人換新眼目不比那些胡牽亂扯忽離忽遇滿紙才人淑女子建文君紅娘小玉等通共熟套之舊稿我師意為何如\侧批{余代空空道人答曰不獨破愁醒盹且有大益}空空道人聽如此說思忖半晌將這石頭\侧批{本名}記再\侧批{這空空道人也太小心了想亦世之一腐儒耳}檢閱
\批注[x=650pt, y=100pt, height=7]{開卷一篇立意真打破歷來小說窠臼閱其茟則是莊子離騷之亞}\批注[x=750pt, y=100pt, height=7]{斯亦太過}

一遍因見上面雖有些指奸責佞貶惡誅邪之語\侧批{亦断不可少}亦非傷時罵世之㫖\侧批{要緊句}及至君仁臣良父慈子孝凡倫常所關之處皆是稱功頌德眷眷無窮實非別書之可比雖其中大㫖談情亦不過實録其事又非假擬妄稱\侧批{要緊句}一味滛邀艶約私訂偷盟之可比因毫不干涉時世\侧批{要緊句}方從頭至尾抄録回來問世傳奇因空見色由色生情傳情入色自色悟空遂易名為情僧改石頭記為情僧録至吳玉峰題曰紅樓夢東魯孔梅溪則題曰風月寶鑑後因曹雪芹于悼紅軒中披閱十載增刪五次纂成目録分出章回則題曰金陵十二釵並題一絶云
\批注[x=400pt, y=100pt, height=7]{雪芹舊有風月寳鑑之書乃其弟棠村序也今棠村已逝余覩新懷舊故仍因之}

\begin{段落}[indent=3]
滿紙荒唐言\空格[2] 一把辛酸淚\\
都云作者痴\空格[2] 誰解其中味\侧批{此是第一首標題詩}
\end{段落}

\批注[x=710pt, y=100pt, height=7]{若云雪芹披閱增刪然後開卷至此這一篇楔子又係誰撰足見作者之茟式狡猾之甚後文如此處者不少這正是作者用画家烟雲糢糊處觀者萬不可被作者瞞蔽了去方是巨眼}
\批注[x=900pt, y=100pt, height=7]{今而後惟愿造化主再出一芹一脂是書何本余二人亦大快遂心于九泉矣甲午八日淚茟}
\批注[x=1000pt, y=100pt, height=7]{能觧者方有辛酸之淚哭成此書壬午除夕書未成芹為淚盡而逝余嘗哭芹淚亦待盡每意覓青埂峯再問石兄余不遇獺頭和尚何悵悵}

至脂硯齋甲戌抄閱再評仍用石頭記

\end{正文}
\end{document}
