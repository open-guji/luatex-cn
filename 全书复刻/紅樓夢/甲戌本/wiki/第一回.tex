出則既明且看石上是何故事按那石上書云\侧批{以石上所記之文}

當日地陷東南這東南一隅有處曰姑蘇\侧批{是金陵}有城曰閶門者最是紅塵中一二等富貴風流之地\侧批{妙極是石頭口氣惜米顛不遇此石}這閶門外有個十里街\侧批{開口失云势利是伏甄封二姓之事}街内有個仁清巷\侧批{又言人情縂為士隱火後伏茟}巷内有個古廟因地方窄狹\侧批{世路寬平者甚少亦鑿}人皆呼作葫蘆廟\侧批{糊塗也故假語從此具焉}廟傍住着一家鄉宦\侧批{不出榮國大族先冩鄉宦小家從小至大是此書章法}姓甄\批注{真後之甄寶玉亦借此音後不注}名費\侧批{廢}字士隱\侧批{托言將真事隱去也}嫡妻封氏\侧批{風因風俗来}情性賢淑深明禮義\侧批{八字正是冩日後之香菱見其根源不凡}家中雖無甚富貴然本地便也推他為望族了\侧批{本地推為望族寕榮則天下推為望族叙事有層落}因這甄士隱禀性恬淡不以功名爲念\侧批{自是羲皇上人便可作是書之朝代年紀矣搃冩香菱根基原與正十二釵無異}每日只以觀花修竹酌酒吟詩為樂到是神仙一流人品只是一件不足如今年已半百膝下無兒\侧批{所謂美中不足也}只有一女乳名英蓮\侧批{設云應怜也}年方三歲

一日炎夏永晝\侧批{熱日無多}士隱於書房閒坐至手倦拋書伏几少憩不覺朦朧睡去夢至一處不辨是何地方忽見那廂來了一僧一道\侧批{是方從青埂峯袖石而来也接得無痕}且行且談

只聽道人問道你携了這蠢物意欲何往那僧笑道你放心如今現有一叚風流公案正該了結這一干風流寃家尚未投胎入世趂此機㑹就將此蠢物夾帶于中使他去經歷經歷那道人道原來近日風流寃孽又將造劫歷世去不成但不知落于何方何處

那僧笑道此事說來好笑竟是千古未聞的罕事只因西方靈河岸上三生石畔\侧批{妙所謂三生石上旧精魂也}\批注{全用幻情之至莫如此今採來壓巷其後可知}有絳\侧批{點紅字}珠\侧批{細思絳珠二字豈非血淚乎}草一株時有赤瑕\侧批{点紅字玉字二}宮神瑛\侧批{單点玉字二}侍者\批注{按瑕字本注玉小赤也又玉有病也以此命名恰極}日以甘露灌溉這絳珠草便得久延歲月後來既受天地精華復得雨露滋餋遂得脫却草胎木質得換人形僅修成箇女體終日逰于離恨天外飢則食密青果為膳渴則飲灌愁海水為湯\侧批{飲食之名竒甚出身履歷更奇甚冩黛玉來歷自與別個不同}只因尚未酬報灌溉之德\侧批{妙極恩怨不清西方尚如此況世之人乎趣甚警甚}故其五衷便鬱結着一叚纒綿不盡之意\批注{以頑石草木為偶實歷盡風月波瀾嚐遍情緣滋味至無可如何始結此木石因果以洩胸中悒鬱古人之一花一石如有意不語不咲能留人此之謂耶}恰近日神瑛侍者凡心偶熾\侧批{縂悔輕舉妄動之意}乘此昌明太平朝世意欲下凡造歴幻\侧批{点幻字}緣已在警幻\侧批{又出一警幻皆大關鍵處}仙子案前掛了號警幻亦曾問及灌溉之情未償趂此到可了結的那絳珠仙子道他是甘露之惠我並無此水可還他既下世為人我也去下世為人但把我一生所有的眼淚還他也償還得過他了\侧批{觀者至此請掩卷思想歷來小說中可曾有此句千古未聞之奇文}\批注{知眼泪還債大都作者一人耳余亦知此意但不能說得出}因此一事就勾出多少風流寃家來\侧批{餘不及一人者盖全部之主惟二玉二人也}賠他們去了結此案

那道人道果是罕聞實未聞有還淚之說想來這一叚故事比歷來風月事故更加鎻碎細膩了那僧道歷來幾個風流人物不過傳其大㮣以及詩詞篇章而已至家庭閨閣中一飲一食總未述記再者大半風月故事不過偷香竊玉暗約私奔而已並不曾將兒女之真情發洩一干人这一人入世其情痴色鬼賢愚不肖者悉與前人傳述不同矣

那道人道趂此你我何不也去下世度脫幾個豈不是一場功德那僧道正合吾意你且同我到警幻仙子宮中將這蠢物交割清楚待這一干風流孽鬼下世已完你我再去如今雖已有一半落塵然猶未全集\侧批{若從頭逐個冩去成何文字石頭記得力處在此丁亥春}道人道既如此便隨你去來

却說甄士隱俱聽得明白但不知所云蠢物係何東西遂不禁上前施禮笑問道二仙師請了那僧道也忙答禮相問士隱因說道適聞仙師所談因果實人世罕聞者但弟子愚濁不能洞悉明白若蒙大開痴頑備細一聞弟子則洗耳諦聽稍能警省亦可免沉淪之苦二僲笑道此乃玄機不可預洩者到那時只不要忘了我二人便可跳出火坑矣士隱聽了不便再問因笑道玄機不可預洩但適云蠢物不知為何或可一見否那僧道若問此物到有一面之緣說着取出遞與士隱士隱接了看時原來是塊鮮明羙玉上面字跡分明鐫着通靈寶玉四字\侧批{凡三四次始出明玉形隱屈之至}後面還有幾行小字正欲細看時那僧便說已到幻境\侧批{又点幻字云書已入幻境矣}便強從手中奪了去與道人竟過一大石牌坊那牌坊上大書四字乃是太虛幻境\侧批{四字可思}兩邊又有一副對聨道是

假作真時真亦假無為有處有還無\侧批{叠用真假有無字妙}

士隱意欲也跟了過去方舉步時忽聽一聲霹靂有若山崩地陷士隱大叫一聲定睛一看只見烈日炎炎芭蕉冉冉\侧批{醒得無痕不落旧套}夢中之事便忘了對半\侧批{妙極若記得便是俗茟了} 又見奶姆正抱了英蓮走來士隱見女兒越發生得粉粧玉琢乖覺可喜便伸手接來抱在懷中闘他頑耍一回又帶至街前看那過㑹的熱鬧方欲進來時只見從那邊來了一僧一道\侧批{所謂萬境都如夢境看也}那僧則癩頭跣足那道則跛足蓬頭\侧批{此門是幻像}瘋瘋顛颠揮霍談笑而至及至到了他門前看見士隱抱着英蓮那僧便哭起來\侧批{竒怪所謂情僧也}又向士隱道施主你把這有命無運累及爹娘之物抱在懷内作甚\批注{八個字屈死多少英雄屈死多少忠臣孝子屈死多少仁人志士屈死多少詞客騷人今又被作者將此一把眼淚灑與閨閣之中見得裙釵尚遭逢此数況天下之男子乎看他所冩開卷之第一個女子便用此二語以訂終身則知託言寓意之㫖誰謂獨寄興于一情字耶}\批注{武侯之三分武穆之二帝二賢之恨及今不盡況今之草芥乎家國君父事有大小之殊其理其運其数則略無差異知運知数者則必諒而後嘆也}士隱聽了知是瘋話也不去採他那僧還說捨我罷捨我罷士隱不奈煩便抱着女兒撤身進去那僧乃指着他大笑口内念了四句言詞道是

慣飬嬌生笑你痴\侧批{為天下父母痴心一哭}
菱花空對雪澌澌\侧批{生不遇時}\侧批{遇又非偶}
好防佳節元霄後\侧批{前後一樣不直云前而云後是諱知者}
便是煙消火滅時\侧批{伏後文}

士隱聽得明白心下猶豫意欲問他們來歷只聽道人說道你我不必同行就此分手各幹营生去罷三劫後\批注{佛以世謂劫凡三十年為一世三劫者想以九十春光寓言也}我在北邙山等你㑹齊了同徃太虛幻境銷號那僧道妙妙妙

說畢二人一去再不見個踪影了士隱心中此時自忖這兩個人必有來歷該試一問如今悔却晚也

這士隱正痴想忽見隔壁\侧批{隔壁二字極細極險記清}葫蘆廟内寄居的一個窮儒姓賈名化\侧批{假話妙}字表時飛\侧批{實非妙}別號雨村\侧批{雨村者村言粗語也言以村粗之言演出一叚假話也}者走了出來這賈雨村原係胡州\侧批{胡謅也}人氏原係詩書仕宦之族因他生于末世\侧批{又冩一末世男子}父母祖宗根基一盡人口衰喪只剩得他一身一口在家鄉無益因進京求取功名再整基業自前歲來此又淹蹇住了蹔寄廟中安身每日賣字作文為生故士隱常與他交接\侧批{又夾冩士隱實是翰林文苑非守錢虜也直灌入慕雅女雅集苦吟詩一回}當下雨村見了士隱施禮陪笑道老先生倚門佇望敢街市上有甚新聞否士隱笑道非也適因小女啼哭引他出來作耍正是無聊之甚兄來得正妙請入小齋一談彼此皆可消此永晝說着便令人送女兒進去自携了雨村來至書房中小童獻茶方談得三五句話忽家人飛報嚴\侧批{炎也炎既來火将至矣}老爺來拜士隱忙的起身謝罪道恕誑駕之罪畧坐即來陪雨村忙起身亦讓道老先生請便晚生乃常造之客稍候何妨說着士隱已出前㕔去了

這里雨村且翻弄書籍解悶忽聽得窗外有女子嗽聲雨村遂起身徃窗外一看原來是一個丫嬛在那里擷花生得儀容不俗眉目清朗\侧批{八字足矣}雖無十分姿色却亦有動人之處\批注{更好這便是真正情理之文可咲近之小說中滿紙羞花閉月等字這是雨村目中又不與後之人相似}雨村不覺看得呆了\侧批{今古窮酸色心最重}那甄家丫嬛擷了花方欲走時猛抬頭見窗内有人敝巾舊服雖是窮貧然生得腰圓背厚面濶口方更兼劍眉星眼直鼻權腮\侧批{是莽操遺容}\批注{最可笑世之小說中凡冩奸人則用鼠耳鷹腮等語}這丫鬟忙轉身廻避心下乃想這人生得這樣雄壯却又這樣繿縷想他定是我家主人常說的什麽賈雨村了每有意幫助週濟只是沒甚機會我家並無這樣貧窮親友想定係此人無疑了怪道又說他必非久困之人如此想不免又回頭兩次\批注{這方是女兒心中意中正文又最恨近之小說中滿紙紅拂紫烟}雨村見他回了頭便自為這女子心中有意于他\侧批{今古窮酸皆會替女婦心中取中自己}便狂喜不禁自為此女子必是個巨眼英豪風塵中之知己也一時小童進來雨村打聽得前面留飯不可久待遂從夾道中自便出門去了士隱待客既散雨村自便也不去再邀

一日早又中秋佳節士隱家宴已畢及又另具一席于書房却自己步月至廟中來邀雨村\侧批{冩士隱愛才好客}原來雨村自那日見了甄家之婢曾回頭顧他兩次自為是個知己便時刻放在心上今又正值中秋不免對月有懷因而口占五言一律云\侧批{這是第一首詩後文香奩閨情皆不落空余謂雪芹撰此書中亦為傳詩之意}

未卜三生願頻添一叚愁
悶來時斂額行去幾回頭
自顧風前影誰堪月下儔
蟾光如有意先上玉人樓

雨村吟罷因又思及平生抱負苦未逢時乃又搔首對天長嘆復高吟一聨云

玉在匱中求善價釵於奩内待時飛\侧批{表過黛玉則緊接上宝釵}
\侧批{前用二玉合傳今用二宝合傳自是書中正眼}

恰至士隱走來聽見笑道雨村兄真抱負不淺也雨村忙笑道豈敢不過偶吟前人之句何敢狂誕至此因問老先生何興至此士隱笑道今夜中秋俗謂團圓之節想尊兄旅寄僧房不無寂寞之感故特具小酌邀兄到敝齋一飲不知可納芹意否雨村聽了並不推辭便笑道既蒙謬愛何敢拂此盛情\侧批{冩雨村豁達氣象不俗}說着便同了士隱復過這邊書院中來

湏臾茶畢早已設下杯盤那羙酒佳餚自不必說二人歸坐先是款斟漫飲次漸談至興濃不覺飛觥限斝起來當時街坊上家家簫管戶戶弦歌當頭一輪明月飛彩凝輝二人愈添豪興酒到杯乾雨村此時已有七八分酒意狂興不禁乃對月寓懷口號一絶云

時逢三五便團圓\侧批{是將發之机}
滿把晴光䕶玉欄\侧批{奸雄心事不覺露出}
天上一輪纔捧出
人間萬姓仰頭看
\批注{這首詩非本㫖不過欲出雨村不得不有者}\批注{用中秋詩起用中秋詩收又用起詩社于秋日所嘆者三春也却用三秋作關鍵}


士隱聽了大呌妙哉吾每謂兄必非久居人下者今所吟之句飛騰之兆已見不日可接履于雲霓之上矣可賀可賀乃親斟一斗為賀\侧批{這個斗字莫作升斗之斗看可咲}雨村因乾過嘆道非晚生酒後狂言若論時尚之學\侧批{四字新而含蓄最廣若必指明則又落套矣}晚生也或可去充數沽名是目今行囊路費一㮣無措神京路遠非賴賣字撰文可能到者士隱不待說完便道兄何不早言愚每有此心但每遇兄時兄並未談及愚故未敢唐突今既及此愚雖不才義利二字却還識得且喜明歲正當大比兄宜作速入都春闈一戰方不負兄之所學也其盤費餘事弟自代為處置尔不枉兄之謬識矣當下即命小童進去速封五十两白銀並兩套冬衣\批注{冩士隱如此豪爽又全無一些粘皮帶骨之氣相愧殺近之讀書假道學矣}又云十九日乃黄道之期兄可即買舟西上待雄飛高舉明冬再晤豈非大快之事耶雨村收了銀衣不過略謝一語並不介意仍是吃酒談笑\侧批{冩雨村真是個英雄}那天已交三鼓二人方散

士隱送雨村去後回房一覺直至紅日三竿方醒\侧批{是宿酒}因思昨夜之事意欲再冩兩封荐書與雨村帶至神京使雨村投謁個仕宦之家為寄足之地\侧批{又週到如此}因使人過去請時那家人去了回來說和尚說賈爺今日五鼓已進京去了也曾留下話與和尚轉達老爺說讀書人不在黄道黑道總以事理爲要不及面辭了\侧批{冩雨村真令人爽快}士隱聽了也只得罷了

真是閒處光陰易過倏忽又是元佳節矣因士隱命家人霍啓\侧批{妙禍起也此因事而命名}抱了英蓮去看社火花燈半夜中霍啟因要小觧便将英蓮放在一家門檻上坐着待他小解完了來抱時那有英蓮的踪影急得霍啟直尋了半夜至天明不見那霍啓也就不敢回來見主人便逃徃他鄉去了那士隱夫婦見女兒一夜不歸便知有些不妥再使幾箇人去尋找回來皆云連音響皆無夫妻二人半世只生此女一旦失落豈不思想因此晝夜啼哭幾乎不曾尋死\批注{喝醒天下父母之痴心}看看一月士隱先就得了一病當時封氏孺人也因思女搆疾日日請醫療病不想這日三月十五葫蘆廟中炸供那些和尚不加小心致使油鍋火逸便燒着窗紙此方人家多用竹籬木璧者\侧批{土俗人風}大抵也因劫數于是接二連三牽五掛四将一條街燒得如火燄山一般\批注{冩出南直召禍之實病}彼時雖有軍民來救那火已成了勢如何救得下去直燒了一夜方漸漸熄去也不知燒了幾家只可憐甄家在隔璧早已燒成一片瓦礫場了只有他夫婦並几個家人的性命不曾傷了急得士隱惟跌足長嘆而已只得與妻子商議且到田庄上去安身偏值近年水旱不收鼠盜蜂起無非搶粮奪食鼠窃狗偷民不安生因此官兵勦捕難以安身士隱只得将田庄都折變了便携了妻子與兩個丫嬛投他岳丈家去

他岳丈名喚封肅本貫大如州人氏\批注{托言大㮣如此之風俗也}雖是務農家中都還殷實今見女婿這等狼狽而來心中便有些不樂\侧批{所以大㮣之人情如是風俗如是也}幸而士隱還有折變地的銀子未曾用完拿出來托他隨分就價薄置些須房地為後日衣食之計那封肅便半哄半賺些湏與他些薄田朽屋士隱乃讀書之人不慣生理稼穡等事勉強支持了一二年越覺窮了下去封肅每見面時便說些現成話且人前人後又怨他們不善過活只一味好吃懶用等語\侧批{此等人何多之極}士隱知投人不着心中未免悔恨再兼上年驚唬急忿悲痛已傷暮年之人貧病交攻竟漸漸露出那下世的光景來

可巧這日拄了拐掙挫在街前散散心時忽見那邊來了一個跛足道人瘋狂落脫蔴屣鶉衣口内念着几句言詞道是

世人都曉神仙好惟有功名忘不了
古今将相在何方荒塚一堆草沒了
世人都曉神仙好只有金銀忘不了
終朝只恨聚無多及到多時眼閉了
世人都曉神仙好只有姣妻忘不了
君生日日說恩情君死又隨人去了
世人都曉神仙好只有兒孫忘不了
痴心父母古來多孝順兒孫誰見了

士隱聽了便迎上來道你滿口說什麽只聽見些好了好了那道人笑道你若果聽見好了二字還算你明白可知世上萬般好便是了了便是好若不了便不好若要好湏是了我這歌兒便名好了歌士隱本是有宿慧的一聞此言心中早已徹悟因笑道且住待我将你這好了歌解註出來何如道人笑道你解你解士隱乃說道

陋室空堂當年笏滿床\侧批{宁榮未有之先}
衰草枯楊曾為歌舞場\侧批{宁榮既敗之後}
蛛絲兒結滿雕梁\侧批{瀟湘館紫芸軒等處}緑紗今又糊在蓬窗上\侧批{雨村等一干新榮暴發之家}
\批注{先說場面忽新忽敗忽麗忽朽已見得反覆不了}
說什麽脂正濃粉正香\侧批{寶釵湘雲一干人}如何兩鬢又成霜\侧批{黛玉晴雯一干人}
昨日黄土隴頭送白骨今宵紅燈帳底卧鴛鴦\侧批{熙鳳一干人}
\批注{一叚妻妾迎新送死倏恩倏愛倏痛倏悲纏綿不了}
金滿箱銀滿箱展眼乞丐人皆謗\侧批{甄玉賈玉一干人}
正嘆他人命不長那知自己歸來喪
\批注{一叚石火光陰悲喜不了風露草霜富貴嗜欲貪婪不了}
訓有方保不定日後\侧批{言父母死後之日}作強梁\侧批{柳湘蓮一干人}擇膏粱誰承望流落在烟花巷
\批注{一叚兒女死後無憑生前空為籌畫計算痴心不了}
因嫌紗帽小致使鎻枷扛\侧批{賈赦雨村一干人}
昨怜破袄寒今嫌紫蟒長\侧批{賈蘭賈菌一干人}
\批注{一叚功名陞黜無時強奪苦争喜惧不了}
亂烘烘你方唱罷我登場\侧批{縂收}
\批注{縂收古今億兆痴人共歴幻場此幻事擾擾紛紛無日可了}
反認他鄉是故鄉\侧批{太虛幻境青埂峯一並結住}
甚荒唐\侧批{語雖舊句用于此妥極是極}到頭來都是為他人作嫁衣裳\侧批{苟能如此便能了得}
\批注{此等歌謠原不宜太雅恐其不能通俗故只此便妙極其說得痛切處又非一味俗語可到}

那瘋跛道人聽了指掌笑道解得切解得切士隱便笑一聲走罷\侧批{如聞如見}\批注{走罷二字真懸崖撒手若個能行}將道人肩上搭連搶了過來背着竟不回家同了瘋道人飄飄而去

當下烘動街坊衆人當作一件新文傳說封氏聞得此信哭個死去活來只得與父親商議遣人各處訪尋那討音信無奈何少不得依靠着他父母度日幸而身邊還有兩個舊日的丫嬛伏侍主僕三人日夜做些個針線發賣帮着父親用度那封肅雖然日日報怨也無可奈何了

這日那甄家的大丫嬛在門前買線忽聽得街上喝道之聲衆人都說新太爺到任丫嬛于是隱在門内看時只見軍牢快手一對一對的過去俄而大轎内抬着一個烏帽猩袍的官府過去\侧批{雨村別来無恙否可賀可賀}\批注{所謂乱烘烘你方唱罷我登場是也}丫嬛到發個怔自思這官好面善到像在那里見過的于是進入房中也就丟過不在心上\侧批{是無兒女之情故有夫人之分}至晚間正該歇息之時忽聽一片聲打的門响許多人亂嚷說本府太爺差人來傳人問話封肅聽了唬得目瞪口呆不知有何禍事