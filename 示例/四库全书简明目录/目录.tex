\documentclass[SiKuMuLu]{guji}

% \禁用分页裁剪
\无标点模式
\setmainfont{TW-Kai}

\title{欽定四庫全書簡明目錄}

\chapter{經部 易類\\卷一}

\begin{document}
\begin{正文}

\印章[
  page=1,
  opacity=0.7,
  color=black,
  xshift=5.3cm,
  yshift=6.7cm,
  width=12.9cm
]{文渊阁宝印.png}

欽定四庫全書簡明目錄卷一

\条目[1]{經部一}
\条目[2]{易類}
    
《子夏易傳》十一卷

\注{舊本題卜子夏撰,實後人輾轉依托,非其原書。然唐、宋以來,流傳已久,今仍錄冠易類之首。凡托名之書,仍從其所托之時代,《漢書·藝文志》例也。}

\按{謹按:唐徐堅《初學記》以太宗御制升列歷代之前,蓋尊尊之大義。焦竑《國史經籍志》,朱彞尊《經義考》並踵前規。臣等編摩《四庫》,初亦恭錄
\单抬《御定易經通注》
\单抬《御纂周易折中》
\平抬《御纂周易述義》,弁冕諸經。仰蒙\\
指示,命冠於}

\注{國朝著述之首,俾尊卑有序,而時代不淆。\平抬 聖度謙沖,酌中立憲,實為千古之大公。謹恪遵\平抬 彞訓,仍託始於《子夏易傳》。並發凡於此,著《四庫》之通例}
\按{焉。}

《周易鄭康成注》一卷

\注{漢鄭玄撰。原本散佚,此本乃宋末王應麟采諸書所引,裒合而成。}

\按{前代佚書而後人重編者,如有所竄改,則從重編之時代。如全輯舊文者,則仍從原書之時代。故此書雖宋人所輯,而列於漢代之中。後皆仿此。}

\end{正文}
\end{document}
